\chapter{Instrumented routines}\label{cha:InstrumentedRoutines}

\section{Instrumented MPI routines}\label{sec:MPIinstrumentedroutines}

These are the instrumented MPI routines in the \TRACE package:

\begin{itemize}
\item MPI\_Init
\item MPI\_Init\_thread\footnotemark[1]
\item MPI\_Finalize
\item MPI\_Bsend
\item MPI\_Ssend
\item MPI\_Rsend
\item MPI\_Send
\item MPI\_Bsend\_init
\item MPI\_Ssend\_init
\item MPI\_Rsend\_init
\item MPI\_Send\_init
\item MPI\_Ibsend
\item MPI\_Issend
\item MPI\_Irsend
\item MPI\_Isend
\item MPI\_Recv
\item MPI\_Irecv
\item MPI\_Recv\_init
\item MPI\_Reduce
\item MPI\_Reduce\_scatter
\item MPI\_Allreduce
\item MPI\_Barrier
\item MPI\_Cancel
\item MPI\_Test
\item MPI\_Wait
\item MPI\_Waitall
\item MPI\_Waitany
\item MPI\_Waitsome
\item MPI\_Bcast
\item MPI\_Alltoall
\item MPI\_Alltoallv
\item MPI\_Allgather
\item MPI\_Allgatherv
\item MPI\_Gather
\item MPI\_Gatherv
\item MPI\_Scatter
\item MPI\_Scatterv
\item MPI\_Comm\_rank
\item MPI\_Comm\_size
\item MPI\_Comm\_create
\item MPI\_Comm\_free
\item MPI\_Comm\_dup
\item MPI\_Comm\_split
\item MPI\_Comm\_spawn
\item MPI\_Comm\_spawn\_multiple
\item MPI\_Cart\_create
\item MPI\_Cart\_sub
\item MPI\_Start
\item MPI\_Startall
\item MPI\_Request\_free
\item MPI\_Scan
\item MPI\_Sendrecv
\item MPI\_Sendrecv\_replace
\item MPI\_File\_open\footnotemark[2]
\item MPI\_File\_close\footnotemark[2]
\item MPI\_File\_read\footnotemark[2]
\item MPI\_File\_read\_all\footnotemark[2]
\item MPI\_File\_write\footnotemark[2]
\item MPI\_File\_write\_all\footnotemark[2]
\item MPI\_File\_read\_at\footnotemark[2]
\item MPI\_File\_read\_at\_all\footnotemark[2]
\item MPI\_File\_write\_at\footnotemark[2]
\item MPI\_File\_write\_at\_all\footnotemark[2]
\item MPI\_Get\footnotemark[3]
\item MPI\_Put\footnotemark[3]
\item MPI\_Win\_create\footnotemark[3]
\item MPI\_Win\_fence\footnotemark[3]
\item MPI\_Win\_free\footnotemark[3]
\item MPI\_Win\_start\footnotemark[3]
\end{itemize}

\footnotetext[1]{The MPI library must support this routine}
\footnotetext[2]{The MPI library must support MPI/IO routines}
\footnotetext[3]{The MPI library must support 1-sided (or RMA -remote memory address-) routines}

\section{Instrumented OpenMP runtimes}\label{sec:OpenMPruntimesinstrumented}

\subsection{Intel compilers - icc, iCC, ifort}

The instrumentation of the Intel OpenMP runtime for versions 8.1 to 10.1 is only available using the \TRACE package based on DynInst library.

These are the instrument routines of the Intel OpenMP runtime functions using DynInst:

\begin{itemize}
\item \_\_kmpc\_fork\_call
\item \_\_kmpc\_barrier
\item \_\_kmpc\_invoke\_task\_func
\item \_\_kmpc\_set\_lock\footnotemark[4]
\item \_\_kmpc\_unset\_lock\footnotemark[4]
\end{itemize}
\footnotetext[4]{The instrumentation of OpenMP locks can be enabled/disabled}

The instrumentation of the Intel OpenMP runtime for version 11.0 to 12.0 is available using the \TRACE package based on the {\tt LD\_PRELOAD} and also the DynInst mechanisms. The instrumented routines include:

\begin{itemize}
\item \_\_kmpc\_fork\_call
\item \_\_kmpc\_barrier
\item \_\_kmpc\_dispatch\_init\_4
\item \_\_kmpc\_dispatch\_init\_8
\item \_\_kmpc\_dispatch\_next\_4
\item \_\_kmpc\_dispatch\_next\_8
\item \_\_kmpc\_dispatch\_fini\_4
\item \_\_kmpc\_dispatch\_fini\_8
\item \_\_kmpc\_single
\item \_\_kmpc\_end\_single
\item \_\_kmpc\_critical\footnotemark[4]
\item \_\_kmpc\_end\_critical\footnotemark[4]
\item omp\_set\_lock\footnotemark[4]
\item omp\_unset\_lock\footnotemark[4]
\item \_\_kmpc\_omp\_task\_alloc
\item \_\_kmpc\_omp\_task\_begin\_if0
\item \_\_kmpc\_omp\_task\_complete\_if0
\item \_\_kmpc\_omp\_taskwait
\end{itemize}

\subsection{IBM compilers - xlc, xlC, xlf}

\TRACE supports IBM OpenMP runtime 1.6.

These are the instrumented routines of the IBM OpenMP runtime:

\begin{itemize}
\item \_xlsmpParallelDoSetup\_TPO
\item \_xlsmpParRegionSetup\_TPO
\item \_xlsmpWSDoSetup\_TPO
\item \_xlsmpBarrier\_TPO
\item \_xlsmpSingleSetup\_TPO
\item \_xlsmpWSSectSetup\_TPO
\item \_xlsmpRelDefaultSLock\footnotemark[4]
\item \_xlsmpGetDefaultSLock\footnotemark[4]
\end{itemize}

\subsection{GNU compilers - gcc, g++, gfortran}

\TRACE supports GNU OpenMP runtime 4.2.

These are the instrumented routines of the GNU OpenMP runtime:

\begin{itemize}
\item GOMP\_parallel\_start
\item GOMP\_parallel\_sections\_start
\item GOMP\_parallel\_end
\item GOMP\_sections\_start
\item GOMP\_sections\_next
\item GOMP\_sections\_end
\item GOMP\_sections\_end\_nowait
\item GOMP\_loop\_end
\item GOMP\_loop\_end\_nowait
\item GOMP\_loop\_static\_start
\item GOMP\_loop\_dynamic\_start
\item GOMP\_loop\_guided\_start
\item GOMP\_loop\_runtime\_start
\item GOMP\_loop\_ordered\_static\_start
\item GOMP\_loop\_ordered\_dynamic\_start
\item GOMP\_loop\_ordered\_guided\_start
\item GOMP\_loop\_ordered\_runtime\_start
\item GOMP\_parallel\_loop\_static\_start
\item GOMP\_parallel\_loop\_dynamic\_start
\item GOMP\_parallel\_loop\_guided\_start
\item GOMP\_parallel\_loop\_runtime\_start
\item GOMP\_loop\_static\_next
\item GOMP\_loop\_dynamic\_next
\item GOMP\_loop\_guided\_next
\item GOMP\_loop\_runtime\_next
\item GOMP\_barrier
\item GOMP\_critical\_name\_enter\footnotemark[4]
\item GOMP\_critical\_name\_exit\footnotemark[4]
\item GOMP\_critical\_enter\footnotemark[4]
\item GOMP\_critical\_exit\footnotemark[4]
\item GOMP\_atomic\_enter\footnotemark[4]
\item GOMP\_atomic\_exit\footnotemark[4]
\item GOMP\_task
\item GOMP\_taskwait
\end{itemize}

\section{Instrumented pthread runtimes}\label{sec:OpenMPruntimesinstrumented}

These are the instrumented routines of the pthread runtime:

\begin{itemize}
\item pthread\_create
\item pthread\_detach
\item pthread\_join
\item pthread\_barrier\_wait
\item pthread\_mutex\_lock
\item pthread\_mutex\_trylock
\item pthread\_mutex\_timedlock
\item pthread\_mutex\_unlock
% pthread_cond_* routines seem to be not instrumentable. the application hangs when instrumenting them
%\item pthread\_cond\_signal
%\item pthread\_cond\_broadcast
%\item pthread\_cond\_wait
%\item pthread\_cond\_timedwait
\item pthread\_rwlock\_rdlock
\item pthread\_rwlock\_tryrdlock
\item pthread\_rwlock\_timedrdlock
\item pthread\_rwlock\_wrlock
\item pthread\_rwlock\_trywrlock
\item pthread\_rwlock\_timedwrlock
\item pthread\_rwlock\_unlock
\end{itemize}

\section{Instrumented CUDA routines}\label{sec:CUDAinstrumentedroutines}

These are the instrumented CUDA routines in the \TRACE package:

\begin{itemize}
\item cudaLaunch
\item cudaConfigureCall
\item	cudaThreadSynchronize
\item	cudaStreamCreate
\item	cudaStreamSynchronize
\item	cudaMemcpy
\item	cudaMemcpyAsync
\item	cudaDeviceReset
\end{itemize}

The CUDA accelerators do not have memory for the tracing buffers, so the tracing buffer resides in the host side.
Typically, the CUDA tracing buffer is flushed at {\tt cudaThreadSynchronize}, {\tt cudaStreamSynchronize} and {\tt cudaMemcpy} calls, so it is possible that the tracing buffer for the device gets filled if no calls to this routines are executed.

\section{Instrumented OpenCL routines}\label{sec:OPENCLinstrumentedroutines}

These are the instrumented OpenCL routines in the \TRACE package:

\begin{itemize}
\item clBuildProgram
\item clCompileProgram
\item clCreateBuffer
\item clCreateCommandQueue
\item clCreateContext
\item clCreateContextFromType
\item clCreateKernel
\item clCreateKernelsInProgram
\item clCreateProgramWithBinary
\item clCreateProgramWithBuiltInKernels
\item clCreateProgramWithSource
\item clCreateSubBuffer
\item clEnqueueBarrierWithWaitList
\item clEnqueueCopyBuffer
\item clEnqueueCopyBufferRect
\item clEnqueueFillBuffer
\item clEnqueueMarkerWithWaitList
\item clEnqueueMapBuffer
\item clEnqueueMigrateMemObjects
\item clEnqueueNativeKernel
\item clEnqueueNDRangeKernel
\item clEnqueueReadBuffer
\item clEnqueueReadBufferRect
\item clEnqueueTask
\item clEnqueueUnmapMemObject
\item clEnqueueWriteBuffer
\item clEnqueueWriteBufferRect
\item clFinish
\item clFlush
\item clLinkProgram
\item clSetKernelArg
\item clWaitForEvents
\end{itemize}

The OpenCL accelerators have small amounts of memory, so the tracing buffer resides in the host side.
Typically, the accelerator tracing buffer is flushed at each {\tt cl\_Finish} call, so it is possible that the tracing buffer for the accelerator gets filled if no calls to this routine are executed.
However if the operated OpenCL command queue is tagged as not Out-of-Order, then flushes will also happen at {\tt clEnqueueReadBuffer}, {\tt clEnqueueReadBufferRect} and {\tt clEnqueueMapBuffer} if their corresponding blocking parameter is set to true.


