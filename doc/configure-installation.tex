\chapter{Configuration, build and installation}\label{cha:Configuration}

\section{Configuration of the instrumentation package}

There are many options to be applied at configuration time for the instrumentation package. We point out here some of the relevant options, sorted alphabetically. To get the whole list run {\tt configure --help}. Options can be enabled or disabled. To enable them use --enable-X or --with-X= (depending on which option is available), to disable them use --disable-X or --without-X.

\begin{itemize}
	\item {\tt --enable-merge-in-trace} \\
	Embed the merging process in the tracing library so the final tracefile can be generated automatically from the application run.
	\item {\tt --enable-parallel-merge} \\
	Build the parallel mergers (mpimpi2prv/mpimpi2dim) based on MPI.
	\item {\tt --enable-posix-clock} \\
	Use POSIX clock (clock\_gettime call) instead of low level timing routines. Use this option if the system where you install the instrumentation package modifies the frequency of its processors at runtime.
	\item {\tt --enable-single-mpi-lib} \\
	Produces a single instrumentation library for MPI that contains both Fortran and C wrappers. Applications that call the MPI library from both C and Fortran languages need this flag to be enabled.
	\item {\tt --enable-spu-write} \\
	Enable direct write operations to disk from SPUs in CELL machines avoiding the usage of DMA transfers. The write mechanism is very slow compared with the original behavior.
	\item {\tt --enable-sampling} \\
	Enable PAPI sampling support.
	\item {\tt --enable-pmapi} \\
	Enable PMAPI library to gather CPU performance counters. PMAPI is a base package installed in AIX systems since version 5.2.
	\item {\tt --enable-cuda} \\
	Enable support for tracing CUDA calls on nVidia hardware.
	\item {\tt --enable-openmp} \\
	Enable support for tracing OpenMP on IBM and GNU runtimes. The IBM runtime instrumentation is only available for PowerPC systems.
	\item {\tt --enable-smpss} \\
	Enable support for tracing SMP-superscalar.
	\item {\tt --enable-nanos} \\
	Enable support for tracing Nanos run-time.
	\item {\tt --enable-pthread} \\
	Enable support for tracing pthread library calls.
	\item {\tt --enable-upc} \\
	Enable support for tracing UPC run-time.
	\item {\tt --enable-xml} \\
	Enable support for XML configuration (not available on BG/L and BG/P systems).
	\item {\tt --enable-xmltest} \\
	Do not try to compile and run a test LIBXML program.
	\item {\tt --enable-doc} \\
	Generates this documentation.
	\item {\tt --prefix=DIR} \\
	Location where the installation will be placed. After issuing {\tt make install} you will find under DIR the entries {\tt lib/}, {\tt include/}, {\tt share/} and {\tt bin/} containing everything needed to run the instrumentation package.
	\item {\tt --with-mpi=DIR} \\
	Specify the location of an MPI installation to be used for the instrumentation package.
	\item {\tt --with-binary-type=OPTION} \\
	Available options are: 32, 64 and default. Specifies the type of memory address model when compiling (32bit or 64bit).
	\item {\tt --with-mpi-name-mangling=OPTION} \\
	Available options are: 0u, 1u, 2u, upcase and auto. Choose the Fortran name decoration (0, 1 or 2 underscores) for MPI symbols. Let OPTION be auto to automatically detect the name mangling.
	\item {\tt --with-pacx=DIR} \\
	Specify where PACX communication library can be find.
	\item {\tt --with-unwind=DIR} \\
	Specify where to find Unwind libraries and includes. This library is used to get callstack information on IA64 and Intel x86-64 on Linux.
	\item {\tt --with-papi=DIR} \\
	Specify where to find PAPI libraries and includes. PAPI is used to gather performance counters.
	\item {\tt --with-bfd=DIR} \\
	Specify where to find the Binary File Descriptor package. In conjunction with libiberty, it is used to translate addresses into source code locations.
	\item {\tt --with-liberty=DIR} \\
	Specify where to find the libiberty package. In conjunction with Binary File Descriptor, it is used to translate addresses into source code locations.
  \item {\tt --with-dyninst=DIR} \\
  Specify the installation location for the DynInst package. \TRACE also requires the DWARF package {\tt --with-dwarf=DIR} when using DynInst.
\end{itemize}

\section{Build}

To build the instrumentation package, just issue {\tt make} after the configuration.

\section{Installation}

To install the instrumentation package in the directory chosen at configure step (through {\tt --prefix} option), issue {\tt make install}.

\section{Examples of configuration on different machines}

All commands given here are given as an example to configure and install the package, you may need to tune them properly (i.e., choose the appropriate directories for packages and so).  These examples assume that you are using a sh/bash shell, you must adequate them if you use other shells (like csh/tcsh).

\subsection{Bluegene (L and P variants)}

Configuration command:

\graybox{{\tt ./configure --prefix=/homec/jzam11/jzam1128/aplic/extrae/2.2.0 --with-papi=/homec/jzam11/jzam1128/aplic/papi/4.1.2.1 --with-bfd=/bgsys/local/gcc/gnu-linux\_4.3.2/powerpc-linux-gnu/powerpc-bgp-linux --with-liberty=/bgsys/local/gcc/gnu-linux\_4.3.2/powerpc-bgp-linux --with-mpi=/bgsys/drivers/ppcfloor/comm --without-unwind --without-dyninst}}


Build and installation commands:

\graybox{{\tt make\\
make install}}

\subsection{AIX}

Some extensions of \TRACE do not work properly (nanos, SMPss and OpenMP) on AIX. In addition, if using IBM MPI (aka POE) the make will complain when generating the parallel merge if the main compiler is not xlc/xlC. So, you can either change the compiler or disable the parallel merge at compile step. Also, command {\tt ar} can complain if 64bit binaries are generated. It's a good idea to run make with OBJECT\_MODE=64 set to avoid this.

\subsubsection{Compiling the 32bit package using the IBM compilers}

Configuration command:

\graybox{{\tt CC=xlc CXX=xlC ./configure --prefix=PREFIX --disable-nanos --disable-smpss --disable-openmp --with-binary-type=32 --without-unwind --enable-pmapi --without-dyninst}}

Build and installation commands:

\graybox{{\tt make\\
make install}}

\subsubsection{Compiling the 64bit package without the parallel merge}

Configuration command:

\graybox{{\tt ./configure --prefix=PREFIX --disable-nanos --disable-smpss --disable-openmp --disable-parallel-merge --with-binary-type=64 --without-unwind --enable-pmapi --without-dyninst}}

Build and installation commands:

\graybox{{\tt OBJECT\_MODE=64 make\\
make install}}


\subsection{Linux}

\subsubsection{Compiling using default binary type using MPICH, OpenMP and PAPI}

Configuration command:

\graybox{{\tt ./configure --prefix=PREFIX --with-mpi=/home/harald/aplic/mpich/1.2.7 --with-papi=/usr/local/papi --enable-openmp}}

Build and installation commands:

\graybox{{\tt make\\
make install}}

\subsubsection{Compiling 32bit package in a 32/64bit mixed environment}

Configuration command:

\graybox{{\tt ./configure --prefix=PREFIX --with-mpi=/opt/osshpc/mpich-mx --with-papi=/gpfs/apps/PAPI/3.6.2-970mp --with-binary-type=32 --with-unwind=\$HOME/aplic/unwind/1.0.1/32 --with-elf=/usr --with-dwarf=/usr --with-dyninst=\$HOME/aplic/dyninst/7.0.1/32}}

Build and installation commands:

\graybox{{\tt make \\
make install}}

\subsubsection{Compiling 64bit package in a 32/64bit mixed environment}

Configuration command:

\graybox{{\tt ./configure --prefix=PREFIX --with-mpi=/opt/osshpc/mpich-mx --with-papi=/gpfs/apps/PAPI/3.6.2-970mp --with-binary-type=64  --with-unwind=\$HOME/aplic/unwind/1.0.1/64 --with-elf=/usr --with-dwarf=/usr --with-dyninst=\$HOME/aplic/dyninst/7.0.1/64}}

Build and installation commands:

\graybox{{\tt make \\
make install}}

\subsubsection{Compiling using default binary type using OpenMPI and PACX}

Configuration command:

\graybox{{\tt ./configure --prefix=PREFIX --with-mpi=/home/harald/aplic/openmpi/1.3.1 --with-pacx=/home/harald/aplic/pacx/07.2009-openmpi --without-papi --without-unwind --without-dyninst}}

Build and installation commands:

\graybox{{\tt make\\
make install}}

\subsubsection{Compiling using default binary type, using OpenMPI, DynInst and libunwind}

\graybox{{\tt ./configure --prefix=PREFIX --with-mpi=/home/harald/aplic/openmpi/1.3.1 --with-dyninst=/home/harald/dyninst/7.0.1 --with-dwarf=/usr --with-elf=/usr --with-unwind=/home/harald/aplic/unwind/1.0.1 --without-papi}}

Build and installation commands:

\graybox{{\tt make\\
make install}}

\subsubsection{Compiling on CRAY XT5 for 64bit package and adding sampling}

Notice the "--disable-xmltest". As the backends programs cannot be run in the frontent, we skip running the XML test. Also using a local installation of libunwind.

Configuration command:

\graybox{{\tt CC=cc CFLAGS='-O3 -g' LDFLAGS='-O3 -g' CXX=CC CXXFLAGS='-O3 -g' ./configure --with-mpi=/opt/cray/mpt/4.0.0/xt/seastar/mpich2-gnu --with-binary-type=64 --with-xml-prefix=/sw/xt5/libxml2/2.7.6/sles10.1\_gnu4.1.2 --disable-xmltest --with-bfd=/opt/cray/cce/7.1.5/cray-binutils --with-liberty=/opt/cray/cce/7.1.5/cray-binutils --enable-sampling --enable-shared=no --prefix=PREFIX --with-papi=/opt/xt-tools/papi/3.7.2/v23 --with-unwind=/ccs/home/user/lib}}

Build and installation commands:

\graybox{{\tt make\\
make install}}

\subsubsection{Compiling on a Power CELL processor using Linux}

\graybox{{\tt ./configure --with-mpi=/opt/openmpi/ppc32 --without-unwind --without-dyninst --without-papi --prefix=/gpfs/data/apps/CEPBATOOLS/extrae/2.2.0/openmpi/32 --with-binary-type=32}}

Build and installation commands:

\graybox{{\tt make\\
make install}}

\subsubsection{Compiling in a Slurm/MOAB environment with support for MPICH2}

Configuration command:

\graybox{{\tt export MP\_IMPL=anl2 ./configure --prefix=PREFIX\\--with-mpi=/gpfs/apps/MPICH2/mx/1.0.8p1..3/32\\--with-papi=/gpfs/apps/PAPI/3.6.2-970mp --with-binary-type=64}}

Build and installation commands:

\graybox{{\tt make\\
make install}}

\subsubsection{Compiling in a environment with IBM compilers and POE}

Configuration command:

\graybox{{\tt CC=xlc CXX=xlC ./configure --prefix=PREFIX --with-mpi=/opt/ibmhpc/ppe.poe}}

Build and installation commands:

\graybox{{\tt make\\
make install}}

\subsubsection{Compiling in a environment with GNU compilers and POE}

Configuration command:

\graybox{{\tt ./configure --prefix=PREFIX --with-mpi=/opt/ibmhpc/ppe.poe}}

Build and installation commands:

\graybox{{\tt MP\_COMPILER=gcc make\\
make install}}

\section{Knowing how a package was configured}

If you are interested on knowing how an \TRACE package was configured execute the following command after setting {\tt EXTRAE\_HOME} to the base location of an installation

\graybox{{\tt \$\{EXTRAE\_HOME\}/etc/configured.sh}}

this command will show the configure command itself and the location of some dependencies of the instrumentation package.
