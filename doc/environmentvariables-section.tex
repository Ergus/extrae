\chapter{Environment variables}\label{cha:EnvVar}

Although \TRACE is configured through an XML file (which is pointed by the {\tt MPTRACE\_CONFIG\_FILE}), it also supports minimal configuration to be done via environment variables for those systems that do not have the library responsible for parsing the XML files ({\em i.e.,} libxml2).

This appendix presents the environment variables the \TRACE package uses if {\tt MPTRACE\_CONFIG\_FILE} is not set and a description. For those environment variable that refer to XML 'enabled' attributes ({\em i.e.}, that can be set to "yes" or "no") are considered to be enabled if their value are defined to 1.
 
\begin{table}[!ht]
\centerline {
\begin{tabular}{| l | l |}
  \hline
  {\bf Environment variable} & {\bf Description}\\
  \hline
  MPTRACE\_BUFFER\_SIZE & Set the number of records that the instrumentation buffer can\\
                        &  hold before flushing them\\
  \hline
  MPTRACE\_CIRCULAR\_BUFFER & \\
  \hline
  MPTRACE\_COUNTERS & See section \ref{subsec:ProcessorPerformanceCounters}. Just one set can be defined. Counters (in PAPI)\\
                    & groups (in PMAPI) are given separated by commas.\\
  \hline
  MPTRACE\_CONTROL\_FILE & \\
  \hline
  MPTRACE\_CONTROL\_GLOPS & \\
  \hline
  MPTRACE\_CONTROL\_TIME & \\
  \hline
  MPTRACE\_DIR & Specifies where temporal files will be created during\\
               & instrumentation\\
  \hline
  MPITRACE\_DISABLE\_MPI & Disable MPI instrumentation\\
  \hline
  MPITRACE\_DISABLE\_OMP & Disable OpenMP instrumentation\\
  \hline
  MPTRACE\_FILE\_SIZE & Set the maximum size (in Mbytes) for the intermediate trace file\\
  \hline
  MPTRACE\_FUNCTIONS & \\
  \hline
  MPTRACE\_FUNCTIONS\_MAX\_DEPTH & {\em (deprecated)} \\
  \hline
  MPTRACE\_FUNCTIONS\_COUNTERS\_ON & Specify if the performance counters should be collected when a\\
                                   & user function event is emitted\\
  \hline
  MPTRACE\_FINAL\_DIR & Specifies where files will be stored when the application ends\\
  \hline
  MPTRACE\_GATHER\_MPITS & Gather intermediate trace files into a single directory\\
  & (\em{this is only available when instrumenting MPI applications})\\
  \hline
  MPITRACE\_HOME & Points where the \TRACE is installed.\\
  \hline
  MPITRACE\_INITIAL\_MODE & Choose whether the instrumentation runs in in {\tt detail} or\\
                          & in {\tt burst} mode\\
  \hline
  MPITRACE\_BURST\_THRESHOLD & Specify the threshold time to filter running bursts\\
  \hline
  MPTRACE\_MINIMUM\_TIME & Specify the minimum amount of instrumentation time\\
  \hline
  MPITRACE\_MPI\_CALLER & Choose which MPI calling routines should be dumped into the\\
                        & tracefile\\
  \hline
  MPITRACE\_MPI\_COUNTERS\_ON & Set to 1 if MPI must report performace counter values\\
  \hline 
  MPITRACE\_MPI\_STATISTICS & Set to 1 if basic MPI statistics must be collected in burst mode\\
  \hline
  MPITRACE\_NETWORK\_COUNTERS & Set to 1 to dump network performance counters at flush points\\
                              & {\em (Only available in systems with Myrinet GM/MX networks)}\\
  \hline
  MPITRACE\_OMP\_COUNTERS\_ON & Set to 1 if OpenMP must report performance counters values. \\
  \hline
  MPTRACE\_OMP\_LOCKS & Set to 1 if locks have to be instrumented\\
  \hline 
  MPITRACE\_ON & Enables instrumentation\\
  \hline
  MPITRACE\_PROGRAM\_NAME & Specify the prefix of the resulting intermediate trace files\\ 
  \hline
  MPITRACE\_RUSAGE & Instrumentation emits resource usage at flush points if set to 1\\
  \hline
  MPTRACE\_SPU\_DMA\_CHANNEL & Choose the SPU-PPU dma communication channel\\
  \hline
  MPTRACE\_SPU\_BUFFER\_SIZE & Set the buffer size of the SPU side\\
  \hline 
  MPTRACE\_SPU\_FILE\_SIZE & Set the maximum size for the SPU side (default: 5Mbytes)\\
  \hline
  MPITRACE\_TRACE\_TYPE & Choose whether the resulting tracefiles are intended for Paraver or Dimemas\\
  \hline
\end{tabular}
}
\caption{Set of environment variables available to configure \TRACE}
\label{tab:EnvironmentVariables}
\end{table}

