\documentclass[twoside,a4,english,11pt]{book}

\usepackage{fontenc}
\usepackage[latin9]{inputenc}
\usepackage{url}
\usepackage{epsfig} % permet posar figures amb format .eps
\usepackage{subfigure} % permet fer subfigures
\usepackage{setspace} % permet canviar l'espai interlinial
\usepackage{tocbibind} % posa la bibliografia a la TOC
\usepackage{a4wide}
\usepackage{fullpage}
\usepackage{verbatim}
\usepackage{color,calc}
\usepackage{pdflscape}
\usepackage{fancyvrb}
\usepackage{amsfonts}
\definecolor{lightgray}{rgb}{0.75,0.75,0.75}

\def\graybox#1{
  \medskip
  \colorbox{lightgray}{
    \begin{minipage}[t]{0.95\textwidth}
    #1
    \end{minipage}
   }
}

\urlstyle{leostyle}

\makeatletter
\def\url@leostyle{
  \@ifundefined{selectfont}{\def\UrlFont{\sf}}{\def\UrlFont{\small\ttfamily}}}
\usepackage{babel}
\makeatother

% \setcounter{tocdepth}{0} % Canviar el deepness de la table of contents
\pagestyle{plain}

% DEFINITIONS!
% DEFINITIONS!
% DEFINITIONS!

\newcommand{\TRACE}{{\sf {E}xtrae}\ }
\newcommand{\PARAVER}{{\sf Paraver}\ }
\input{version}

% DOCUMENT!
% DOCUMENT!
% DOCUMENT!


\begin{document}

\pagenumbering{roman}

% \onehalfspacing
% \doublespacing

\title{\TRACE \\
       User guide manual\\
       for version \TRACEVERSION}
\author{
tools@bsc.es
}

\maketitle
\tableofcontents
\listoffigures
\listoftables

\input{common-macros}

\chapter{Quick start guide}\label{cha:QuickStart}

\section{The instrumentation package}

\TRACE is a dynamic instrumentation package to trace programs compiled and run with the shared memory model (like OpenMP and pthreads), the message passing (MPI) programming model or both programming models (different MPI processes using OpenMP or pthrads within each MPI process). \TRACE generates trace files that can be latter visualized with \PARAVER.

The package is distributed in compressed tar format (e.g., extrae.tar.gz).  To unpack it, execute from the desired target directory the following command line :

\begin{verbatim}
               gunzip -c extrae.tar.gz | tar -xvf -
\end{verbatim}

The unpacking process will create different directories on the current directory (see table \ref{tab:PackageDescription}).

\begin{table}[!ht]
\centerline {
\begin{tabular}{| l | l |}
\hline
{\bf Directory}     & {\bf Contents} \\
\hline
bin           & Contains the binary files of the \TRACE tool.\\
\hline
etc           & Contains some scripts to set up environment variables and the\\
              & \TRACE internal files.\\
\hline
lib           & Contains the \TRACE tool libraries.\\
\hline
share/man     & Contains the \TRACE manual entries.\\
\hline
share/doc     & Contains the \TRACE manuals (pdf, ps and html versions).\\
\hline
share/example & Contains examples to illustrate the \TRACE instrumentation.\\
\hline
\end{tabular}
}
\caption{Package contents description}
\label{tab:PackageDescription}
\end{table}

\section{Quick running}

Once the package has been unpacked, set the {\tt EXTRAE\_HOME} environment variable to the directory where the package was installed. Use the {\tt export} or {\tt setenv} commands to set it up depending on the shell you use.  If you use sh-based shell (like sh, bash, ksh, zsh, ...), issue this command
\begin{verbatim}
               export EXTRAE_HOME=dir
\end{verbatim}
however, if you use csh-based shell (like csh, tcsh), execute the following command
\begin{verbatim}
               setenv EXTRAE_HOME dir
\end{verbatim}
where {\em dir} refers where the \TRACE was installed. Henceforth, all references to the usage of the environment variables will be used following the sh format unless specified.

\TRACE is offered in two different flavors: as a DynInst-based application, or stand-alone application. DynInst is a dynamic instrumentation library that allows the injection of code in a running application without the need to recompile the target application. If the DynInst instrumentation library is not installed, \TRACE also offers different mechanisms to trace applications.

\subsection{Quick running \TRACE - based on DynInst}\label{subsec:RunningTraceDynInst}

\TRACE needs some environment variables to be setup on each session. Issuing the command 

\begin{verbatim}
               source ${EXTRAE_HOME}/etc/extrae.sh
\end{verbatim}

on a sh-based shell, or 

\begin{verbatim}
               source ${EXTRAE_HOME}/etc/extrae.csh
\end{verbatim}

on a csh-based shell will do the work. Then copy the default XML configuration file\footnote{See section \ref{cha:XML} for further details regarding this file} into the working directory by executing

\begin{verbatim}
               cp ${EXTRAE_HOME}/share/example/MPI/mpitrace.xml .
\end{verbatim}

If needed, set the application environment variables as usual (like {\tt OMP\_NUM\_THREADS}, for example), and finally launch the application using the {\tt \$\{EXTRAE\_HOME\}/bin/ompitrace} instrumenter like:

\begin{verbatim}
               ${EXTRAE_HOME}/bin/ompitrace -config mpitrace.xml <program>
\end{verbatim}

where {\tt <program>} is the application binary.

\subsection{Quick running \TRACE - NOT based on DynInst}\label{subsec:RunningTraceNOTDynInst}

\TRACE needs some environment variables to be setup on each session. Issuing the command 

\begin{verbatim}
               source ${EXTRAE_HOME}/etc/extrae.sh
\end{verbatim}

on a sh-based shell, or 

\begin{verbatim}
               source ${EXTRAE_HOME}/etc/extrae.csh
\end{verbatim}

on a csh-based shell will do the work. Then copy the default XML configuration file\footnotemark[1] into the working directory by executing

\begin{verbatim}
               cp ${EXTRAE_HOME}/share/example/MPI/mpitrace.xml .
\end{verbatim}

and export the {EXTRAE\_CONFIG\_FILE} as

\begin{verbatim}
               export EXTRAE_CONFIG_FILE=mpitrace.xml
\end{verbatim}

If needed, set the application environment variables as usual (like {\tt OMP\_NUM\_THREADS}, for example). Just before executing the target application, issue the following command:

\begin{verbatim}
               export LD_PRELOAD=${EXTRAE_HOME}/lib/<lib>
\end{verbatim}

where {\tt <lib>} is

\begin{itemize}
 \item {\tt libmpitrace.so} If the application is just MPI.
 \item {\tt libomptrace.so} If the application is just OpenMP.
 \item {\tt libompitrace.so} If the application is MPI and OpenMP.
\end{itemize}

\section{Quick merging the intermediate traces}

Once the intermediate trace files (*.mpit files) have been created, they have to be merged (using the {\tt mpi2prv} command) in order to generate the final \PARAVER trace file. Execute the following command to proceed with the merge:

\begin{verbatim}
               ${EXTRAE_HOME}/bin/mpi2prv -f TRACE.mpits -o output.prv
\end{verbatim}

The result of the merge process is a \PARAVER tracefile called output.prv. If the -o option is not given, the resulting tracefile is called MPITRACE\_Paraver\_Trace.prv. 


\chapter{Introduction}\label{cha:Introduction}


\TRACE is a dynamic instrumentation package to trace programs compiled and run with the shared memory model (like OpenMP and pthreads), the message passing (MPI) programming model or both programming models (different MPI processes using OpenMP or pthreads within each MPI process). \TRACE generates trace files that can be visualized with \PARAVER.

\TRACE is currently available on different platforms and operating systems: IBM PowerPC running Linux or AIX, and x86 and x86-64 running Linux. It also has been ported to OpenSolaris and FreeBSD.

The combined use of \TRACE and \PARAVER offers an enormous analysis potential, both qualitative and quantitative. With these tools the actual performance bottlenecks of parallel applications can be identified. The microscopic view of the program behavior that the tools provide is very useful to optimize the parallel program performance.

This document tries to give the basic knowledge to use the \TRACE tool. Chapter \ref{cha:Configuration} explains how the package can be configured and installed. Chapter \ref{cha:Examples} explains how to monitor an application to obtain its trace file. At the end of this document there are appendices that include: a Frequent Asked Questions appendix and a list of routines instrumented in the package.

\subsection*{What is the \PARAVER tool?}

\PARAVER is a flexible parallel program visualization and analysis tool based on an easy-to-use Motif GUI. \PARAVER was developed responding to the need of hacing a qualitative global perception of the application behavior by visual inspection and then to be able to focus on the detailed quantitative analysis of the problems. \PARAVER provides a large amount of information useful to decide the points on which to invest the programming effort to optimize an application.

Expressive power, flexibility and the capability of efficiently handling large traces are key features addressed in the design of \PARAVER. The clear and modular structure of \PARAVER plays a significant role towers achieving these targets.

Some \PARAVER features are the support for:
\begin{itemize}
\item Detailed quantitative analysis of program performance,
\item concurrent comparative analysis of several traces,
\item fast analysis of very large traces,
\item support for mixed message passing and shared memory (network of SMPs), and,
\item customizable semantics of the visualized information.
\end{itemize}

One of the main features of \PARAVER is the flexibility to represent traces coming from different environments. Traces are composed of state records, events and communications with associated timestamp. These three elements can be used to build traces that capture the behavior along time of very different kind of systems. The \PARAVER distribution includes, either in its own distribution or as additional packages, the following instrumentation modules:
\begin{enumerate}
\item  Sequential application tracing: it is included in the \PARAVER distribution. It can be used to trace the value of certain variables, procedure invocations, ... in a sequential program.
\item Parallel application tracing: a set of modules are optionally available to capture the activity of parallel applications using shared-memory, message-passing paradigms, or a combination of them.
\item System activity tracing in a multiprogrammed environment: an application to trace processor allocations and process migrations is optionally available in the \PARAVER distribution.
\item Hardware counters tracing: an application to trace the hardware counter values is optionally available in the \PARAVER distribution.
\end{enumerate}

\subsection*{Where the \PARAVER tool can be found?}

The \PARAVER distribution can be found at URL:

\url{http://www.bsc.es/paraver}

\PARAVER binaries are available for Linux/x86, Linux/x86-64 and Linux/ia64, Windows.

In the Documentation Tool section of the aforementioned URL you can find the {\em \PARAVER Reference Manual} and {\em \PARAVER Tutorial} in addition to the documentation for other instrumentation packages.

\TRACE and \PARAVER tools e-mail support is {\bf tools@bsc.es}.

\chapter{Configuration, build and installation}\label{cha:Configuration}

\section{Configuration of the instrumentation package}

There are many options to be applied at configuration time for the instrumentation package. We point out here some of the relevant options, sorted alphabetically. To get the whole list run {\tt configure --help}. Options can be enabled or disabled. To enable them use --enable-X or --with-X= (depending on which option is available), to disable them use --disable-X or --without-X.

\begin{itemize}
	\item {\tt --enable-merge-in-trace} \\
	Embed the merging process in the tracing library so the final tracefile can be generated automatically from the application run.
	\item {\tt --enable-parallel-merge} \\
	Build the parallel mergers (mpimpi2prv/mpimpi2dim) based on MPI.
	\item {\tt --enable-posix-clock} \\
	Use POSIX clock (clock\_gettime call) instead of low level timing routines. Use this option if the system where you install the instrumentation package modifies the frequency of its processors at runtime.
	\item {\tt --enable-single-mpi-lib} \\
	Produces a single instrumentation library for MPI that contains both Fortran and C wrappers. Applications that call the MPI library from both C and Fortran languages need this flag to be enabled.
	\item {\tt --enable-spu-write} \\
	Enable direct write operations to disk from SPUs in CELL machines avoiding the usage of DMA transfers. The write mechanism is very slow compared with the original behavior.
	\item {\tt --enable-sampling} \\
	Enable PAPI sampling support.
	\item {\tt --enable-pmapi} \\
	Enable PMAPI library to gather CPU performance counters. PMAPI is a base package installed in AIX systems since version 5.2.
	\item {\tt --enable-openmp} \\
	Enable support for tracing OpenMP on IBM, GNU and Intel runtimes. The IBM runtime instrumentation is only available for Linux/PowerPC systems.
	\item {\tt --enable-openmp-gnu} \\
	Enable support for tracing OpenMP on GNU runtime.
	\item {\tt --enable-openmp-intel} \\
	Enable support for tracing OpenMP on Intel runtime.
	\item {\tt --enable-openmp-ibm} \\
	Enable support for tracing OpenMP IBM runtime. The IBM runtime instrumentation is only available for Linux/PowerPC systems.
    \item {\tt --enable-openmp-ompt} \\
    Enables support for tracing OpenMP runtimes through the OMPT specification. \textbf{NOTE:} enabling this option disables the regular instrumentation system available through \texttt{--enable-openmp-ibm}, \texttt{--enable-openmp-intel} and \texttt{--enable-openmp-gnu}.
	\item {\tt --enable-smpss} \\
	Enable support for tracing SMP-superscalar.
	\item {\tt --enable-nanos} \\
	Enable support for tracing Nanos run-time.
	\item {\tt --enable-pthread} \\
	Enable support for tracing pthread library calls.
	\item {\tt --enable-upc} \\
	Enable support for tracing UPC run-time.
	\item {\tt --enable-xml} \\
	Enable support for XML configuration (not available on BG/L, BG/P and BG/Q systems).
	\item {\tt --enable-xmltest} \\
	Do not try to compile and run a test LIBXML program.
	\item {\tt --enable-doc} \\
	Generates this documentation.
	\item {\tt --prefix=DIR} \\
	Location where the installation will be placed. After issuing {\tt make install} you will find under DIR the entries {\tt lib/}, {\tt include/}, {\tt share/} and {\tt bin/} containing everything needed to run the instrumentation package.
	\item {\tt --with-mpi=DIR} \\
	Specify the location of an MPI installation to be used for the instrumentation package. This flag is mandatory.
	\item {\tt --with-binary-type=OPTION} \\
	Available options are: 32, 64 and default. Specifies the type of memory address model when compiling (32bit or 64bit).
	\item {\tt --with-boost=DIR} \\
	Specify the location of the BOOST package. This package is required when using the DynInst instrumentation with versions newer than 7.0.1.
	\item {\tt --with-mpi-name-mangling=OPTION} \\
	Available options are: 0u, 1u, 2u, upcase and auto. Choose the Fortran name decoration (0, 1 or 2 underscores) for MPI symbols. Let OPTION be auto to automatically detect the name mangling.
	\item {\tt --with-pacx=DIR} \\
	Specify where PACX communication library can be find.
	\item {\tt --with-unwind=DIR} \\
	Specify where to find Unwind libraries and includes. This library is used to get callstack information on several architectures (including IA64 and Intel x86-64). This flag is mandatory.
	\item {\tt --with-papi=DIR} \\
	Specify where to find PAPI libraries and includes. PAPI is used to gather performance counters. This flag is mandatory.
	\item {\tt --with-bfd=DIR} \\
	Specify where to find the Binary File Descriptor package. In conjunction with libiberty, it is used to translate addresses into source code locations.
	\item {\tt --with-liberty=DIR} \\
	Specify where to find the libiberty package. In conjunction with Binary File Descriptor, it is used to translate addresses into source code locations.
	\item {\tt --with-dyninst=DIR} \\
	Specify the installation location for the DynInst package. \TRACE also requires the DWARF package {\tt --with-dwarf=DIR} when using DynInst. Also, newer versions of DynInst (versions after 7.0.1) require the BOOST package {\tt --with-boost}. This flag is mandatory. Requires a working installation of a C++ compiler.
	\item {\tt --with-cuda=DIR} \\
	Enable support for tracing CUDA calls on nVidia hardware and needs to point to the CUDA SDK installation path. This instrumentation is only valid in binaries that use the shared version of the CUDA library. Interposition has to be done through the {\tt LD\_PRELOAD} mechanism. It is superseded by {\tt --with-cupti=DIR} which also supports instrmentation for static binaries.
	\item {\tt --with-cupti=DIR} \\
	Specify the location of the CUPTI libraries. CUPTI is used to instrument CUDA calls, and supersedes the {\tt --with-cuda}, although it still requires {\tt --with-cuda}.
    \item {\tt --with-java=DIR} \\
    Specify the location of JAVA development kit.
\end{itemize}

\section{Build}

To build the instrumentation package, just issue {\tt make} after the configuration.

\section{Installation}

To install the instrumentation package in the directory chosen at configure step (through {\tt --prefix} option), issue {\tt make install}.

\section{Check}

The \TRACE package contains some consistency checks. The aim of such checks is to determine whether a functionality is operative in the target (installation) environment and/or check whether the development of \TRACE has introduced any misbehavior. To run the checks, just issue {\tt make check} after the installation. Please, notice that checks are meant to be run in the machine that the configure script was run, thus the results of the checks on machines with back-end nodes different to front-end nodes (like BG/* systems) are not representative at all.

\section{Examples of configuration on different machines}

All commands given here are given as an example to configure and install the package, you may need to tune them properly (i.e., choose the appropriate directories for packages and so).  These examples assume that you are using a sh/bash shell, you must adequate them if you use other shells (like csh/tcsh).

\subsection{Bluegene (L and P variants)}

Configuration command:

\graybox{{\tt ./configure --prefix=/homec/jzam11/jzam1128/aplic/extrae/2.2.0 --with-papi=/homec/jzam11/jzam1128/aplic/papi/4.1.2.1 --with-bfd=/bgsys/local/gcc/gnu-linux\_4.3.2/powerpc-linux-gnu/powerpc-bgp-linux --with-liberty=/bgsys/local/gcc/gnu-linux\_4.3.2/powerpc-bgp-linux --with-mpi=/bgsys/drivers/ppcfloor/comm --without-unwind --without-dyninst}}

Build and installation commands:

\graybox{{\tt make\\
make install}}

\subsection{BlueGene/Q}

To enable parsing the XML configuration file, the libxml2 must be installed. As of the time of writing this user guide, we have been only able to install the static version of the library in a BG/Q machine, so take this into consideration if you install the libxml2 in the system.  Similarly, the binutils package (responsible for translating application addresses into source code locations) that is available in the system may not be properly installed and we suggest installing the binutils from the source code using the BG/Q cross-compiler. Regarding the cross-compilers, we have found that using the IBM XL compilers may require using the XL libraries when generating the final application binary with Extrae, so we would suggest using the GNU cross-compilers ({\tt /bgsys/drivers/ppcfloor/gnu-linux/bin/powerpc64-bgq-linux-*}).

If you want to add libxml2 and binutils support into Extrae, your configuration command may resemble to:

\graybox{{\tt./configure --prefix=/homec/jzam11/jzam1128/aplic/juqueen/extrae/2.2.1 --with-mpi=/bgsys/drivers/ppcfloor/comm/gcc --without-unwind --without-dyninst --disable-openmp --disable-pthread\\ --with-libz=/bgsys/local/zlib/v1.2.5\\ --with-papi=/usr/local/UNITE/packages/papi/5.0.1\\ --with-xml-prefix=/homec/jzam11/jzam1128/aplic/juqueen/libxml2-gcc\\ --with-binutils=/homec/jzam11/jzam1128/aplic/juqueen/binutils-gcc\\ --enable-merge-in-trace}}

Otherwise, if you do not want to add support for the libxml2 library, your configuration may look like this:

\graybox{{\tt ./configure --prefix=/homec/jzam11/jzam1128/aplic/juqueen/extrae/2.2.1 --with-mpi=/bgsys/drivers/ppcfloor/comm/gcc --without-unwind --without-dyninst --disable-openmp --disable-pthread\\ --with-libz=/bgsys/local/zlib/v1.2.5\\ --with-papi=/usr/local/UNITE/packages/papi/5.0.1 --disable-xml}}

In any situation, the build and installation commands are:

\graybox{{\tt make\\
make install}}

\subsection{AIX}

Some extensions of \TRACE do not work properly (nanos, SMPss and OpenMP) on AIX. In addition, if using IBM MPI (aka POE) the make will complain when generating the parallel merge if the main compiler is not xlc/xlC. So, you can either change the compiler or disable the parallel merge at compile step. Also, command {\tt ar} can complain if 64bit binaries are generated. It's a good idea to run make with OBJECT\_MODE=64 set to avoid this.

\subsubsection{Compiling the 32bit package using the IBM compilers}

Configuration command:

\graybox{{\tt CC=xlc CXX=xlC ./configure --prefix=PREFIX --disable-nanos --disable-smpss --disable-openmp --with-binary-type=32 --without-unwind --enable-pmapi --without-dyninst --with-mpi=/usr/lpp/ppe.poe}}

Build and installation commands:

\graybox{{\tt make\\
make install}}

\subsubsection{Compiling the 64bit package without the parallel merge}

Configuration command:

\graybox{{\tt ./configure --prefix=PREFIX --disable-nanos --disable-smpss --disable-openmp --disable-parallel-merge --with-binary-type=64 --without-unwind --enable-pmapi --without-dyninst --with-mpi=/usr/lpp/ppe.poe}}

Build and installation commands:

\graybox{{\tt OBJECT\_MODE=64 make\\
make install}}


\subsection{Linux}

\subsubsection{Compiling using default binary type using MPICH, OpenMP and PAPI}

Configuration command:

\graybox{{\tt ./configure --prefix=PREFIX --with-mpi=/home/harald/aplic/mpich/1.2.7 --with-papi=/usr/local/papi --enable-openmp --without-dyninst --without-unwind}}

Build and installation commands:

\graybox{{\tt make\\
make install}}

\subsubsection{Compiling 32bit package in a 32/64bit mixed environment}

Configuration command:

\graybox{{\tt ./configure --prefix=PREFIX --with-mpi=/opt/osshpc/mpich-mx --with-papi=/gpfs/apps/PAPI/3.6.2-970mp --with-binary-type=32 --with-unwind=\$HOME/aplic/unwind/1.0.1/32 --with-elf=/usr --with-dwarf=/usr --with-dyninst=\$HOME/aplic/dyninst/7.0.1/32}}

Build and installation commands:

\graybox{{\tt make \\
make install}}

\subsubsection{Compiling 64bit package in a 32/64bit mixed environment}

Configuration command:

\graybox{{\tt ./configure --prefix=PREFIX --with-mpi=/opt/osshpc/mpich-mx --with-papi=/gpfs/apps/PAPI/3.6.2-970mp --with-binary-type=64  --with-unwind=\$HOME/aplic/unwind/1.0.1/64 --with-elf=/usr --with-dwarf=/usr --with-dyninst=\$HOME/aplic/dyninst/7.0.1/64}}

Build and installation commands:

\graybox{{\tt make \\
make install}}

\subsubsection{Compiling using default binary type using OpenMPI and PACX}

Configuration command:

\graybox{{\tt ./configure --prefix=PREFIX --with-mpi=/home/harald/aplic/openmpi/1.3.1 --with-pacx=/home/harald/aplic/pacx/07.2009-openmpi --without-papi --without-unwind --without-dyninst}}

Build and installation commands:

\graybox{{\tt make\\
make install}}

\subsubsection{Compiling using default binary type, using OpenMPI, DynInst and libunwind}

Configuration command:

\graybox{{\tt ./configure --prefix=PREFIX --with-mpi=/home/harald/aplic/openmpi/1.3.1 --with-dyninst=/home/harald/dyninst/7.0.1 --with-dwarf=/usr\\--with-elf=/usr --with-unwind=/home/harald/aplic/unwind/1.0.1\\--without-papi}}

Build and installation commands:

\graybox{{\tt make\\
make install}}

\subsubsection{Compiling on CRAY XT5 for 64bit package and adding sampling}

Notice the "--disable-xmltest". As backends programs cannot be run in the frontend, we skip running the XML test. Also using a local installation of libunwind.

Configuration command:

\graybox{{\tt CC=cc CFLAGS='-O3 -g' LDFLAGS='-O3 -g' CXX=CC CXXFLAGS='-O3 -g' ./configure --with-mpi=/opt/cray/mpt/4.0.0/xt/seastar/mpich2-gnu --with-binary-type=64 --with-xml-prefix=/sw/xt5/libxml2/2.7.6/sles10.1\_gnu4.1.2 --disable-xmltest --with-bfd=/opt/cray/cce/7.1.5/cray-binutils --with-liberty=/opt/cray/cce/7.1.5/cray-binutils --enable-sampling --enable-shared=no --prefix=PREFIX --with-papi=/opt/xt-tools/papi/3.7.2/v23 --with-unwind=/ccs/home/user/lib --without-dyninst}}

Build and installation commands:

\graybox{{\tt make\\
make install}}

\subsubsection{Compiling for the Intel MIC accelerator / Xeon Phi}

The Intel MIC accelerators (also codenamed KnightsFerry - KNF and KnightsCorner - KNC) or Xeon Phi processors are not binary compatible with the host (even if it is an Intel x86 or x86/64 chip), thus the Extrae package must be compiled specially for the accelerator (twice if you want Extrae for the host). While the host configuration and installation has been shown before, in order to compile Extrae for the accelerator you must configure Extrae like:

\graybox{{\tt ./configure --with-mpi=/opt/intel/impi/4.1.0.024/mic --without-dyninst --without-papi --without-unwind --disable-xml --disable-posix-clock --with-libz=/opt/extrae/zlib-mic --host=x86\_64-suse-linux-gnu --prefix=/home/Computational/harald/extrae-mic --enable-mic\\CFLAGS="-O -mmic -I/usr/include" CC=icc CXX=icpc\\MPICC=/opt/intel/impi/4.1.0.024/mic/bin/mpiicc}}

To compile it, just issue:

\graybox{{\tt make\\
make install}}

\subsubsection{Compiling on a Power CELL processor using Linux}

Configuration command:

\graybox{{\tt ./configure --with-mpi=/opt/openmpi/ppc32 --without-unwind --without-dyninst --without-papi --prefix=/gpfs/data/apps/CEPBATOOLS/extrae/2.2.0/openmpi/32 --with-binary-type=32}}

Build and installation commands:

\graybox{{\tt make\\
make install}}

\subsubsection{Compiling on a ARM based processor machine using Linux}

If using the GNU toolchain to compile the library, we suggest at least using version 4.6.2 because of its enhaced in this architecture.

Configuration command:

\graybox{{\tt CC=/gpfs/APPS/BIN/GCC-4.6.2/bin/gcc-4.6.2 ./configure --prefix=/gpfs/CEPBATOOLS/extrae/2.2.0\\--with-unwind=/gpfs/CEPBATOOLS/libunwind/1.0.1-git\\--with-papi=/gpfs/CEPBATOOLS/papi/4.2.0 --with-mpi=/usr --enable-posix-clock --without-dyninst}}

Build and installation commands:

\graybox{{\tt make\\
make install}}

\subsubsection{Compiling in a Slurm/MOAB environment with support for MPICH2}

Configuration command:

\graybox{{\tt export MP\_IMPL=anl2 ./configure --prefix=PREFIX\\--with-mpi=/gpfs/apps/MPICH2/mx/1.0.8p1..3/32\\--with-papi=/gpfs/apps/PAPI/3.6.2-970mp --with-binary-type=64 --without-dyninst --without-unwind}}

Build and installation commands:

\graybox{{\tt make\\
make install}}

\subsubsection{Compiling in a environment with IBM compilers and POE}

Configuration command:

\graybox{{\tt CC=xlc CXX=xlC ./configure --prefix=PREFIX --with-mpi=/opt/ibmhpc/ppe.poe --without-dyninst --without--unwind --without-papi}}

Build and installation commands:

\graybox{{\tt make\\
make install}}

\subsubsection{Compiling in a environment with GNU compilers and POE}

Configuration command:

\graybox{{\tt ./configure --prefix=PREFIX --with-mpi=/opt/ibmhpc/ppe.poe --without-dyninst --without--unwind --without-papi}}

Build and installation commands:

\graybox{{\tt MP\_COMPILER=gcc make\\
make install}}

\subsubsection{Compiling in Hornet / Cray XC40 system}

Configuration command, enabling MPI, PAPI and online analysis over MRNet.

\graybox{{\tt./configure --prefix=/zhome/academic/HLRS/xhp/xhpgl/tools/extrae/intel --with-mpi=/opt/cray/mpt/7.1.2/gni/mpich2-intel/140\\--with-unwind=/zhome/academic/HLRS/xhp/xhpgl/tools/libunwind --without-dyninst --with-papi=/opt/cray/papi/5.3.2.1 --enable-online --with-mrnet=/zhome/academic/HLRS/xhp/xhpgl/tools/mrnet/4.1.0 --with-spectral=/zhome/academic/HLRS/xhp/xhpgl/tools/spectral/3.1 --with-mrnetapp=/zhome/academic/HLRS/xhp/xhpgl/tools/libmrnetapp/1.0.4}}

Build and installation commands:

\graybox{{\tt make\\
make install}}

\section{Knowing how a package was configured}

If you are interested on knowing how an \TRACE package was configured execute the following command after setting {\tt EXTRAE\_HOME} to the base location of an installation

\graybox{{\tt \$\{EXTRAE\_HOME\}/etc/configured.sh}}

this command will show the configure command itself and the location of some dependencies of the instrumentation package.


\chapter{\TRACE XML configuration file}\label{cha:XML}

\TRACE is configured through a XML file that is set through the {\tt EXTRAE\_CONFIG\_FILE} environment variable. The included examples provide several XML files to serve as a basis for the end user. There are four XML files:
\begin{itemize}
 \item {\tt extrae.xml} Exemplifies all the options available to set up in the configuration file. We will discuss below all the sections and options available. It is also available on this document on appendix \ref{cha:wholeXML}.
 \item {\tt extrae\_explained.xml} The same as the above with some comments on each section.
 \item {\tt summarized\_trace\_basic.xml} A small example for gathering information of MPI and OpenMP information with some performace counters and calling information at each MPI call.
 \item {\tt detailed\_trace\_basic.xml} A small example for gathering a summarized information of MPI and OpenMP parallel paradigms.
\end{itemize}

Please note that most of the nodes present in the XML file have an {\tt enabled} attribute that allows turning on and off some parts of the instrumentation mechanism. For example, {\tt <mpi enabled="yes">} means MPI instrumentation is enabled and process all the contained XML subnodes, if any; whether {\tt <mpi enabled="no">} means to skip gathering MPI information and do not process XML subnodes.

Each section points which environment variables could be used if the tracing package lacks XML support. See appendix \ref{cha:EnvVar} for the entire list.

Sometimes the XML tags are used for time selection (duration, for instance). In such tags, the following postfixes can be used: n or ns for nanoseconds, u or us for microseconds, m or ms for milliseconds, s for seconds, M for minutes, H for hours and D for days.

\section{XML Section: Trace configuration}\label{sec:XMLSectionTraceConfiguration}

The basic trace behavior is determined in the first part of the XML and {\bf contains} all of the remaining options. It looks like:

\begin{verbatim}
<?xml version='1.0'?>

<trace enabled="yes"
 home="@sed_MYPREFIXDIR@"
 initial-mode="detail"
 type="paraver"
 xml-parser-id="@sed_XMLID@"
>

< ... other XML nodes ... >

</trace>
\end{verbatim}


The {\tt <?xml version='1.0'?>} is mandatory for all XML files. Don't touch this. The available tunable options are under the {\tt <trace>} node:
\begin{itemize}
 \item {\tt enabled} Set to {\tt "yes"} if you want to generate tracefiles.
 \item {\tt home} Set to where the instrumentation package is installed. Usually it points to the same location that {\tt EXTRAE\_HOME} environment variable.
 \item {\tt initial-mode} Available options
  \begin{itemize}
   \item {\tt detail} Provides detailed information of the tracing.
   \item {\tt bursts} Provides summarized information of the tracing. This mode removes most of the information present in the detailed traces (like OpenMP and MPI calls among others) and only produces information for computation bursts.
  \end{itemize} 
 \item {\tt type} Available options
  \begin{itemize}
   \item {\tt paraver} The intermediate files are meant to generate Paraver tracefiles.
   \item {\tt dimemas} The intermediate files are meant to generate Dimemas tracefiles.
  \end{itemize}
 \item {\tt xml-parser-id} This is used to check whether the XML parsing scheme and the file scheme match or not.
\end{itemize}

\graybox{See {\bf EXTRAE\_ON}, {\bf EXTRAE\_HOME}, {\bf EXTRAE\_INITIAL\_MODE} and {\bf EXTRAE\_TRACE\_TYPE} environment variables in appendix \ref{cha:EnvVar}.}

\section{XML Section: MPI}\label{sec:XMLSectionMPI}

The MPI configuration part is nested in the config file (see section \ref{sec:XMLSectionTraceConfiguration}) and its nodes are the following:

\begin{verbatim}
<mpi enabled="yes">
  <counters enabled="yes" />
</mpi>
\end{verbatim}


MPI calls can gather performance information at the begin and end of MPI calls. To activate this behavior, just set to yes the attribute of the nested {\tt <counters>} node.

\graybox{See {\bf EXTRAE\_DISABLE\_MPI} and {\bf EXTRAE\_MPI\_COUNTERS\_ON} environment variables in appendix \ref{cha:EnvVar}.}

\section{XML Section: PACX}\label{sec:XMLSectionPACX}

The PACX configuration part is nested in the config file (see section \ref{sec:XMLSectionTraceConfiguration}) and its nodes are the following:

\begin{verbatim}
<pacx enabled="yes">
  <counters enabled="yes" />
</pacx>
\end{verbatim}


PACX calls can gather performance information at the begin and end of PACX calls. To activate this behavior, just set to yes the attribute of the nested {\tt <counters>} node.

\graybox{See {\bf EXTRAE\_DISABLE\_PACX} and {\bf EXTRAE\_PACX\_COUNTERS\_ON} environment variables in appendix \ref{cha:EnvVar}.}

\section{XML Section: pthread}\label{sec:XMLSectionOpenMP}

The pthread configuration part is nested in the config file (see section \ref{sec:XMLSectionTraceConfiguration}) and its nodes are the following:

\begin{verbatim}
<pthread enabled="yes">
  <locks enabled="no" />
  <counters enabled="yes" />
</pthread>
\end{verbatim}


The tracing package allows to gather information of some pthread routines. In addition to that, the user can also enable gathering information of locks and also gathering performance counters in all of these routines. This is achieved by modifying the enabled attribute of the {\tt <locks>} and {\tt <counters>}, respectively.

\graybox{See {\bf EXTRAE\_DISABLE\_PTHREAD}, {\bf EXTRAE\_PTHREAD\_LOCKS} and {\bf EXTRAE\_PTHREAD\_COUNTERS\_ON} environment variables in appendix \ref{cha:EnvVar}.}

\section{XML Section: OpenMP}\label{sec:XMLSectionOpenMP}

The OpenMP configuration part is nested in the config file (see section \ref{sec:XMLSectionTraceConfiguration}) and its nodes are the following:

\begin{verbatim}
<openmp enabled="yes">
  <locks enabled="no" />
  <counters enabled="yes" />
</openmp>
\end{verbatim}


The tracing package allows to gather information of some OpenMP runtimes and outlined routines. In addition to that, the user can also enable gathering information of locks and also gathering performance counters in all of these routines. This is achieved by modifying the enabled attribute of the {\tt <locks>} and {\tt <counters>}, respectively.

\graybox{See {\bf EXTRAE\_DISABLE\_OMP}, {\bf EXTRAE\_OMP\_LOCKS} and {\bf EXTRAE\_OMP\_COUNTERS\_ON} environment variables in appendix \ref{cha:EnvVar}.}

\section{XML Section: CELL}\label{sec:XMLcell}

The Cell configuration part is only parsed for tracing packages suited for the Cell architecture, and as the rest of sections it is nested in the config file (see section \ref{sec:XMLSectionTraceConfiguration}). The available nodes only affect the SPE side, and they are:

\begin{verbatim}
<cell enabled="no">
  <spu-file-size enabled="yes">5</spu-file-size>
  <spu-buffer-size enabled="yes">64</spu-buffer-size>
  <spu-dma-channel enabled="yes">2</spu-dma-channel>
</cell>
\end{verbatim}


\begin{itemize}
 \item {\tt spu-file-size} Limits the resulting intermediate trace file for each SPE thread that has been instrumented.
 \item {\tt spu-buffer-size} Specifies the number of events contained in the buffer on the SPE side. Remember that memory is very scarce on the SPE, so setting a high value can exhaust all memory.
 \item {\tt spu-dma-channel} Chooses which {DMA} channel will be used to perform the intermediate trace files transfers to the PPE side.
\end{itemize}

\graybox{See {\bf EXTRAE\_SPU\_FILE\_SIZE}, {\bf EXTRAE\_SPU\_BUFFER\_SIZE} and {\bf EXTRAE\_SPU\_DMA\_CHANNEL} environment variables in appendix \ref{cha:EnvVar}.}

\section{XML Section: Callers}\label{sec:XMLSectionCallers}

\begin{verbatim}
<callers enabled="yes">
  <mpi enabled="yes">1-3</mpi>
  <pacx enabled="no">1-3</pacx>
  <sampling enabled="no">1-5</sampling>
</callers>
\end{verbatim}


Callers are the routine addresses present in the process stack at any given moment during the application run. Callers can be used to link the tracefile with the source code of the application.

The instrumentation library can collect a partial view of those addresses during the instrumentation. Such collected addresses are translated by the merging process if the correspondent parameter is given and the application has been compiled and linked with debug information.

There are three points where the instrumentation can gather this information:

\begin{itemize}
 \item Entry of MPI calls
 \item Entry of PACX calls
 \item Sampling points {\em (if sampling is available in the tracing package)}
 \item Dynamic memory calls (malloc, free, realloc)
\end{itemize}

The user can choose which addresses to save in the trace (starting from 1, which is the closest point to the MPI call or sampling point) specifying several stack levels by separating them by commas or using the hyphen symbol.

\graybox{See {\bf EXTRAE\_MPI\_CALLER} and {\bf EXTRAE\_PACX\_CALLER} environment variables in appendix \ref{cha:EnvVar}.}

\section{XML Section: User functions}\label{sec:XMLSectionUF}

\begin{verbatim}
<user-functions enabled="no" list="/home/bsc41/bsc41273/user-functions.dat">
  <counters enabled="yes" />
</user-functions>
\end{verbatim}



The file contains a list of functions to be instrumented by \TRACE. There are different alternatives to instrument application functions, and some alternatives provides additional flexibility, as a result, the format of the list varies depending of the instrumentation mechanism used:

\begin{itemize}

 \item DynInst\\
  Supports instrumentation of  user functions, outer loops, loops and basic blocks.
  The given list contains the desired function names to be instrumented. After each function name, optionally you can define different basic blocks or loops inside the desired function always by providing different suffixes that are provided after the {\tt +} character. For instance:\\
  \begin{itemize}
  \item To instrument the entry and exit points of foo function just provide the function name ({\tt foo}).
  \item To instrument the entry and exit points of foo function plus the entry and exit points of its outer loop, suffix the function name with {\tt outerloops} ({\em i.e.} {\tt foo+outerloops}).
  \item To instrument the entry and exit points of foo function plus the entry and exit points of its N-th loop function you have to suffix it as {\tt loop\_N}, for instance {\tt foo+loop\_3}.
  \item To instrument the entry and exit points of foo function plus the entry and exit points of its N-th basic block inside the function you have to use the suffix {\tt bb\_N}, for instqance {\tt foo+bb\_5}. In this case, it is also possible to specifically ask for the entry or exit point of the basic block by additionally suffixing {\tt \_s} or {\tt \_e}, respectively.
  \end{itemize}
  Additionally, these options can be added by using comas, as in:\\
  {\tt foo+outerloops,loop\_3,bb\_3\_e,bb\_4\_s,bb\_5}.

  To discover the instrumentable loops and basic blocks of a certain function you can execute the command {\tt \${EXTRAE\_HOME}/bin/extrae -config extrae.xml -decodeBB}, where {\tt extrae.xml} is an \TRACE configuration file that provides a list on the user functions attribute that you want to get the information.
  \item GCC and ICC (through {\tt -finstrument-functions})\\
  GNU and Intel compiler provides a compile and link flag named {\tt -finstrument-functions} that instruments the routines of a source code file that \TRACE can use. To take advantage of this functionality the list of routines must point to a list with the format:
	{\tt hexadecimal address\#function name}
  where hexadecimal address refers to the hexadecimal address of the function in the binary file (obtained throug the {\tt nm binary} and function name is the name of the function to be instrumented. For instance to instrument the routine {\tt pi\_kernel} from the {\tt pi} binary we execute nm as follows:
  \begin{verbatim}
    # nm -a pi | grep pi_kernel
    00000000004005ed T pi_kernel
  \end{verbatim}
  and add {\tt 00000000004005ed \# pi\_kernel} into the function list.
\end{itemize}

The {\tt exclude-automatic-functions} attribute is used only by the DynInst instrumenter. By setting this attribute to {\tt yes} the instrumenter will avoid automatically instrumenting the routines that either call OpenMP outlined routines (i.e. routines with OpenMP pragmas) or call CUDA kernels.

Finally, in order to gather performance counters in these functions and also in those instrumented using the {\tt extrae\_user\_function} API call, the node {\tt counters} has to be enabled.

\graybox{See {\bf EXTRAE\_FUNCTIONS} environment variable in appendix \ref{cha:EnvVar}.}

\section{XML Section: Performance counters}\label{sec:XMLSectionPerformanceCounters}

The instrumentation library can be compiled with support for collecting performance metrics of different components available on the system. These components include:

\begin{itemize}
 \item Processor performance counters. Such access is granted by PAPI\footnote{More information available on their website \url{http://icl.cs.utk.edu/papi}. \TRACE requires PAPI 3.x at least.} or PMAPI\footnote{PMAPI is only available for AIX operating system, and it is on the base operating system since AIX5.3. \TRACE requires AIX 5.3 at least.}
 \item Network performance counters. {\em (Only available in systems with Myrinet GM/MX networks).}
 \item Operating system accounts.
\end{itemize}

Here is an example of the counters section in the XML configuration file:

\begin{verbatim}
<counters enabled="yes">
  <cpu enabled="yes" starting-set-distribution="1">
    <set enabled="yes" domain="all" changeat-time="5s">
      PAPI_TOT_INS,PAPI_TOT_CYC,PAPI_L1_DCM
      <sampling enabled="yes" period="100000000">PAPI_TOT_CYC</sampling>
    </set>
    <set enabled="yes" domain="user" changeat-globalops="5">
      PAPI_TOT_INS,PAPI_TOT_CYC,PAPI_FP_INS
    </set>
  </cpu>
  <network enabled="yes" />
  <resource-usage enabled="yes" />
</counters>
\end{verbatim}


\graybox{See {\bf EXTRAE\_COUNTERS}, {\bf EXTRAE\_NETWORK\_COUNTERS} and {\bf EXTRAE\_RUSAGE} environment variables in appendix \ref{cha:EnvVar}.}

\subsection{Processor performance counters}\label{subsec:ProcessorPerformanceCounters}

Processor performance counters are configured in the {\tt <cpu>} nodes. The user can configure many sets in the {\tt <cpu>} node using the {\tt <set>} node, but just one set will be used at any given time in a specific task. The {\tt <cpu>} node supports the {\tt starting-set-distribution} attribute with the following accepted values:

\begin{itemize}
 \item {\tt number} ({\em in range 1..N, where N is the number of configured sets}) All tasks will start using the set specified by number.
 \item {\tt block} Each task will start using the given sets distributed in blocks ({\em i.e.}, if two sets are defined and there are four running tasks: tasks 1 and 2 will use set 1, and tasks 3 and 4 will use set 2).
 \item {\tt cyclic} Each task will start using the given sets distributed cyclically ({\em i.e.}, if two sets are defined and there are four running tasks: tasks 1 and 3 will use, and tasks 2 and 4 will use set 2).
 \item {\tt random} Each task will start using a random set, and also calls either to {\tt Extrae\_next\_hwc\_set} or {\tt Extrae\_previous\_hwc\_set} will change to a random set.
\end{itemize}

Each set contain a list of performance counters to be gathered at different instrumentation points (see sections \ref{sec:XMLSectionMPI}, \ref{sec:XMLSectionOpenMP} and \ref{sec:XMLSectionUF}). If the tracing library is compiled to support PAPI, performance counters must be given using the canonical name (like PAPI\_TOT\_CYC and PAPI\_L1\_DCM), or the PAPI code in hexadecimal format (like 8000003b and 80000000, respectively)\footnote{Some architectures do not allow grouping some performance counters in the same set.}. If the tracing library is compiled to support PMAPI, only one group identifier can be given per set\footnote{Each group contains several performance counters} and can be either the group name (like pm\_basic and pm\_hpmcount1) or the group number (like 6 and 22, respectively). 

In the given example (which refers to PAPI support in the tracing library) two sets are defined. First set will read {PAPI\_TOT\_INS} (total instructions), {PAPI\_TOT\_CYC} (total cycles) and {PAPI\_L1\_DCM} (1st level cache misses). Second set is configured to obtain {PAPI\_TOT\_INS} (total instructions), {PAPI\_TOT\_CYC} (total cycles) and {PAPI\_FP\_INS} (floating point instructions).

Additionally, if the underlying performance library supports sampling mechanisms, each set can be configured to gather information (see section \ref{sec:XMLSectionCallers}) each time the specified counter reaches a specific value. The counter that is used for sampling must be present in the set. In the given example, the first set is enabled to gather sampling information every 100M cycles.

Furthermore, performance counters can be configured to report accounting on different basis depending on the {\tt domain} attribute specified on each set. Available options are
\begin{itemize}
 \item {\tt kernel} Only counts events ocurred when the application is running in kernel mode.
 \item {\tt user} Only counts events ocurred when the application is running in user-space mode.
 \item {\tt all} Counts events independently of the application running mode.
\end{itemize}

In the given example, first set is configured to count all the events ocurred, while the second one only counts those events ocurred when the application is running in user-space mode.

Finally, the instrumentation can change the active set in a manual and an automatic fashion. To change the active set manually see {\tt Extrae\_previous\_hwc\_set} and {\tt Extrae\_next\_hwc\_set} API calls in \ref{sec:BasicAPI}. To change automatically the active set two options are allowed: based on time and based on application code. The former mechanism requires adding the attribute {\tt changeat-time} and specify the minimum time to hold the set. The latter requires adding the attribute {\tt changeat-globalops} with a value. The tracing library will automatically change the active set when the application has executed as many MPI global operations as selected in that attribute. When In any case, if either attribute is set to zero, then the set will not me changed automatically.

\subsection{Network performance counters}\label{subsec:NetworkPerformanceCounters}

Network performance counters are only available on systems with Myrinet GM/MX networks and they are fixed depending on the firmware used. Other systems, like BG/* may provide some network performance counters, but they are accessed through the PAPI interface (see section \ref{sec:XMLSectionPerformanceCounters} and {PAPI} documentation).

If {\tt <network>} is enabled the network performance counters appear at the end of the application run, giving a summary for the whole run.

\subsection{Operating system accounting}\label{subsec:OperatingSystemAccounting}

Operating system accounting is obtained through the getrusage(2) system call when {\tt <resource-usage>} is enabled. As network performance counters, they appear at the end of the application run, giving a summary for the whole run.

\section{XML Section: Storage management}\label{sec:XMLSectionStorage}

The instrumentation packages can be instructed on what/where/how produce the intermediate trace files. These are the available options:

\begin{verbatim}
<storage enabled="no">
  <trace-prefix enabled="yes">TRACE</trace-prefix>
  <size enabled="no">5</size>
  <temporal-directory enabled="yes">/scratch</temporal-directory>
  <final-directory enabled="yes">/gpfs/scratch/bsc41/bsc41273</final-directory>
</storage>
\end{verbatim}


Such options refer to:

\begin{itemize}
 \item {\tt trace-prefix} Sets the intermediate trace file prefix. Its default value is {TRACE}.
 \item {\tt size} Let the user restrict the maximum size (in megabytes) of each resulting intermediate trace file\footnote{This check is done each time the buffer is flushed, so the resulting size of the intermediate trace file depends also on the number of elements contained in the tracing buffer (see section \ref{sec:XMLSectionBuffer}).}.
 \item {\tt temporal-directory} Where the intermediate trace files will be stored during the execution of the application. By default they are stored in the current directory. If the directory does not exist, the instrumentation will try to make it.\\
 \item {\tt final-directory} Where the intermediate trace files will be stored once the execution has been finished. By default they are stored in the current directory. If the directory does not exist, the instrumentation will try to make it.\\
 \item {\tt gather-mpits} If the system does not provide a global filesystem the resulting trace files will be distributed among the computation nodes. Turning on this option will use the underlying communication mechanism ({MPI}) to gather all the intermediate trace files into the root node.
\end{itemize}

\graybox{See {\bf EXTRAE\_PROGRAM\_NAME}, {\bf EXTRAE\_FILE\_SIZE}, {\bf EXTRAE\_DIR}, {\bf EXTRAE\_FINAL\_DIR} and {\bf EXTRAE\_GATHER\_MPITS} environment variables in appendix \ref{cha:EnvVar}.}

\section{XML Section: Buffer management}\label{sec:XMLSectionBuffer}

Modify the buffer management entry to tune the tracing buffer behavior.

\begin{verbatim}
<buffer enabled="yes">
  <size enabled="yes">150000</size>
  <circular enabled="no" />
</buffer>
\end{verbatim}


By, default (even if the enabled attribute is "no") the tracing buffer is set to 500k events (see section \ref{sec:XMLcell} for further information of buffer in the CELL). If {\tt <size>} is enabled the tracing buffer will be set to the number of events indicated by this node. If the circular option is enabled, the buffer will be created as a circular buffer and the buffer will be dumped only once with the last events generated by the tracing package.

\graybox{See {\bf EXTRAE\_BUFFER\_SIZE} environment variable in appendix \ref{cha:EnvVar}.}

\section{XML Section: Trace control}\label{sec:XMLSectionTraceControl}

\begin{verbatim}
<trace-control enabled="yes">
  <file enabled="no" frequency="5M">/gpfs/scratch/bsc41/bsc41273/control</file>
  <global-ops enabled="no">10</global-ops>
  <remote-control enabled="yes">
    <mrnet enabled="yes" target="150" analysis="spectral" start-after="30">
      <clustering max_tasks="26" max_points="8000"/>
      <spectral min_seen="1" max_periods="0" num_iters="3" signals="DurBurst,InMPI"/>
    </mrnet>
  </remote-control>
</trace-control> 
\end{verbatim}


This section groups together a set of options to limit/reduce the final trace size. There are three mechanisms which are based on file existance, global operations executed and external remote control procedures.

Regarding the {\tt file}, the application starts with the tracing disabled, and it is turned on when a control file is created. Use the property {\tt frequency} to choose at which frequency this check must be done. If not supplied, it will be checked every 100 global operations on MPI\_COMM\_WORLD.

If the {\tt global-ops} tag is enabled, the instrumentation package begins disabled and starts the tracing when the given number of global operations on MPI\_COMM\_WORLD has been executed.

The {\tt remote-control} tag section allows to configure some external mechanisms to automatically control the tracing. Currently, there is only one option which is built on top of MRNet and it is based on clustering and spectral analysis to generate a small yet representative trace.

These are the options in the {\tt mrnet} tag:

\begin{itemize}
 \item {\bf target}: the approximate requested size for the final trace (in Mb).
 \item {\bf analysis}: one between {\tt clustering} and {\tt spectral}.
 \item {\bf start-after}: number of seconds before the first analysis starts.
\end{itemize}

The {\tt clustering} tag configures the clustering analysis parameters:
\begin{itemize}
 \item {\bf max\_tasks}: maximum number of tasks to get samples from.
 \item {\bf max\_points}: maximum number of points to cluster.
\end{itemize}

The {\tt spectral} tag section configures the spectral analysis parameters:
\begin{itemize}
 \item {\bf min\_seen}: minimum times a given type of period has to be seen to trace a sample
 \item {\bf max\_periods}: maximum number of representative periods to trace. 0 equals to unlimited.
 \item {\bf num\_iters}: number of iterations to trace for every representative period found.
 \item {signals}: performance signals used to analyze the application. If not specified, {\tt DurBurst} is used by default.
\end{itemize}

A signal can be used to terminate the tracing when using the remote control. Available values can be only USR1/USR2 Some MPI implementations handle one of those, so check first which is available to you. Set in tag {\tt signal} the signal code you want to use.

\graybox{See {\bf EXTRAE\_CONTROL\_FILE}, {\bf EXTRAE\_CONTROL\_GLOPS}, {\bf EXTRAE\_CONTROL\_TIME} environment variables in appendix \ref{cha:EnvVar}.}

\section{XML Section: Bursts}\label{sec:XMLSectionBursts}

\begin{verbatim}
<bursts enabled="no">
  <threshold enabled="yes">500u</threshold>
  <mpi-statistics enabled="yes" />
</bursts>
\end{verbatim}


If the user enables this option, the instrumentation library will just emit information of computation bursts ({\em i.e.}, not does not trace {MPI} calls, {OpenMP} runtime, and so on) when the current mode (through initial-mode in \ref{sec:XMLSectionTraceConfiguration}) is set to {\tt bursts}. The library will discard all those computation bursts that last less than the selected threshold.

In addition to that, when the tracing library is running in burst mode, it computes some statistics of MPI and PACX activity. Such statistics can be dumped in the tracefile by enabling {\tt mpi-statistics} and {\tt pacx-statistics} respectively.

\graybox{See {\bf EXTRAE\_INITIAL\_MODE}, {\bf EXTRAE\_BURST\_THRESHOLD}, {\bf EXTRAE\_MPI\_STATISTICS} and {\bf EXTRAE\_PACX\_STATISTICS} environment variables in appendix \ref{cha:EnvVar}.}

\section{XML Section: Others}\label{sec:XMLSectionOthers}

\begin{verbatim}
<others enabled="yes">
  <minimum-time enabled="no">10M</minimum-time>
  <finalize-on-signal enabled="yes" 
    SIGUSR1="yes" SIGUSR2="yes" SIGINT="yes"
    SIGQUIT="yes" SIGTERM="yes" SIGXCPU="yes"
    SIGFPE="yes" SIGSEGV="yes" SIGABRT="yes"
  />
  <flush-sampling-buffer-at-instrumentation-point enabled="yes" />
</others>
\end{verbatim}


This section contains other configuration details that do not fit in the previous sections. Right now, there is only one option available and it is devoted to tell the instrumentation package the minimum instrumentation time. To enable it, set {\tt enabled} to "yes" and set the minimum time within the {\tt minimum-time} tag.

\section{XML Section: Sampling}\label{sec:XMLSectionSampling}

\begin{verbatim}
<sampling enabled="no" type="default" period="50m" />
\end{verbatim}


This section configures the time-based sampling capabilities. Every sample contains processor performance counters (if enabled in section \ref{subsec:ProcessorPerformanceCounters} and either PAPI or PMAPI are referred at configure time) and callstack information (if enabled in section \ref{sec:XMLSectionCallers} and proper dependencies are set at configure time).

This section contains two attributes besides {\tt enabled}. These are
\begin{itemize}
 \item {\bf type}: determines which timer domain is used (see {\tt man 2 setitimer} or {\tt man 3p setitimer} for further information on time domains). Available options are: {\tt real} (which is also the {\tt default} value, {\tt virtual} and {\tt prof} (which use the SIGALRM, SIGVTALRM and SIGPROF respectively). The default timing accumulates real time, but only issues samples at master thread. To let all the threads to collect samples, the type must be {\tt virtual} or {\tt prof}.
 \item {\bf period}: specifies the sampling periodicity. In the example above, samples are gathered every 50ms.
 \item {\bf variability}: specifies the variability to the sampling periodicity. Such variability is calculated through the {\tt random()} system call and then is added to the periodicity. In the given example, the variability is set to 10ms, thus the final sampling period ranges from 45 to 55ms.
\end{itemize}

\graybox{See {\bf EXTRAE\_SAMPLING\_PERIOD}, {\bf EXTRAE\_SAMPLING\_VARIABILITY}, {\bf EXTRAE\_SAMPLING\_CLOCKTYPE} and {\bf EXTRAE\_SAMPLING\_CALLER} environment variables in appendix \ref{cha:EnvVar}.}

\section{XML Section: CUDA}\label{sec:XMLSectionCUDA}

\begin{verbatim}
<cuda enabled="yes" />
\end{verbatim}


This section indicates whether the CUDA calls should be instrumented or not. If {\tt enabled} is set to yes, CUDA calls will be instrumented, otherwise they will not be instrumented.

\section{XML Section: OpenCL}\label{sec:XMLSectionOPENCL}

\begin{verbatim}
<opencl enabled="yes" />
\end{verbatim}


This section indicates whether the OpenCL calls should be instrumented or not. If {\tt enabled} is set to yes, Opencl calls will be instrumented, otherwise they will not be instrumented.

\section{XML Section: Input/Output}\label{sec:XMLSectionIO}

\begin{verbatim}
<input-output enabled="no" />
\end{verbatim}
\begin{verbatim}
<input-output enabled="no" />
\end{verbatim}


This section indicates whether I/O calls ({\tt read} and {\tt write}) are meant to be instrumented. If {\tt enabled} is set to yes, the aforementioned calls will be instrumented, otherwise they will not be instrumented.

{\bf Note:} This is an experimental feature, and needs to be enabled at configure time usint the {\tt --enable-instrument-io} option.

{\bf Warning:} This option seems to intefere with the instrumentation of the GNU and Intel OpenMP runtimes, and the issues haven't been solved yet.

\section{XML Section: Dynamic memory}\label{sec:XMLSectionDynamicMemory}

\begin{verbatim}
<dynamic-memory enabled="no" />
\end{verbatim}


This section indicates whether dynamic memory calls ({\tt malloc}, {\tt free}, {\tt realloc}) are meant to be instrumented. If {\tt enabled} is set to yes, the aforementioned calls will be instrumented, otherwise they will not be instrumented.
This section allows deciding whether allocation and free-related memory calls shall be instrumented.
Additionally, the configuration can also indicate whether allocation calls should be instrumented if the requested memory size surpasses a given threshold (32768 bytes, in the example).

{\bf Note:} This is an experimental feature, and needs to be enabled at configure time usint the {\tt --enable-instrument-dynamic-memory} option.

{\bf Warning:} This option seems to intefere with the instrumentation of the Intel OpenMP runtime, and the issues haven't been solved yet.

\section{XML Section: Merge}\label{sec:XMLSectionMerge}

\begin{verbatim}
<merge enabled="yes" 
  synchronization="default"
  binary="mpi_ping"
  tree-fan-out="16"
  max-memory="512"
  joint-states="yes"
  keep-mpits="yes"
  sort-addresses="no"
>
  mpi_ping.prv 
</merge>
\end{verbatim}


If this section is enabled and the instrumentation packaged is configured to support this, the merge process will be automatically invoked after the application run. The merge process will use all the resources devoted to run the application.

In the example given, the leaf of this node will be used as the tracefile name ({\tt mpi\_ping.prv} in this example). Current available options for the merge process are given as attribute of the {\tt <merge>} node and they are:

\begin{itemize}
 \item {\tt synchronization}: which can be set to {\tt default}, {\tt node}, {\tt task}, {\tt no}. This determines how task clocks will be synchronized ({\em default is node}).
 \item {\tt binary}: points to the binary that is being executed. It will be used to translate gathered addresses (MPI callers, sampling points and user functions) into source code references.
 \item {\tt tree-fan-out}: {\em only for MPI executions} sets the tree-based topology to run the merger in a parallel fashion.
 \item {\tt max-memory}: limits the intermediate merging process to run up to the specified limit (in MBytes).
 \item {\tt joint-states}: which can be set to {\tt yes}, {\tt no}. Determines if the resulting Paraver tracefile will split or join equal consecutive states ({\em default is yes}).
 \item {\tt keep-mpits}: whether to keep the intermediate tracefiles after performing the merge.
 \item {\tt sort-addresses}: whether to sort all addresses that refer to the source code (enabled by default).
 \item {\tt overwrite}: set to yes if the new tracefile can overwrite an existing tracefile with the same name. If set to no, then the tracefile will be given a new name using a consecutive id.
\end{itemize}

In Linux systems, the tracing package can take advantage of certain functionalities from the system and can guess the binary name, and from it the tracefile name. In such systems, you can use the following reduced XML section replacing the earlier section.

\begin{verbatim}
<merge enabled="yes" 
  synchronization="default"
  tree-fan-out="16"
  max-memory="512"
  joint-states="yes"
  keep-mpits="yes"
  sort-addresses="yes"
  overwrite="yes"
/>
\end{verbatim}


\graybox{For further references, see chapter \ref{cha:Merging}.}

\section{Using environment variables within the XML file}\label{sec:EnvVars_in_XML}

XML tags and attributes can refer to environment variables that are defined in the environment during the application run. If you want to refer to an environment variable within the XML file, just enclose the name of the variable using the dollar symbol ({\tt \$}), for example: {\tt \$FOO\$}.

Note that the user has to put an specific value or a reference to an environment variable which means that expanding environment variables in text is not allowed as in a regular shell (i.e., the instrumentation package will not convert the follwing text {\tt bar\$FOO\$bar}).



\chapter{\TRACE API}\label{cha:API}

There are two levels of the API in the \TRACE instrumentation package. Basic API refers to the basic functionality provided and includes emitting events, source code tracking, changing instrumentation mode and so. Extended API is an {\em experimental} addition to provide several of the basic API within single and powerful calls using specific data structures.

\section{Basic API}\label{sec:BasicAPI}

The following routines are defined in the {\tt \$\{EXTRAE\_HOME\}/include/extrae\_user\_events.h}. These routines are intended to be called by C/C++ programs. The instrumentation package also provides bindings for Fortran applications. The Fortran API bindings have the same name as the C API but honoring the Fortran compiler function name mangling scheme. The \TRACE constants for Fortran applications can be included from {\tt \$\{EXTRAE\_HOME\}/include/extraef\_user\_events.h}.

\begin{itemize}

 \item {\tt void Extrae\_get\_version (unsigned *major, unsigned *minor, unsigned *revision)}\\
 Returns the version of the underlying \TRACE package. Although an application may be compiled to a specific \TRACE library, by using the appropriate shared library commands, the application may use a different \TRACE library.

 \item {\tt void Extrae\_init (void)}\\
 Initializes the tracing library.\\
 {\bf NOTE:} This routine is called automatically by either {\tt MPI\_Init} or when tracing is performed by the DynInst launcher.

 \item {\tt extrae\_init\_type\_t Extrae\_is\_initialized (void)}\\
 This routine tells whether the instrumentation has been initialized, and if so, also which mechanism was the first to initialize it (regular API, MPI or PACX initialization).

 \item {\tt void Extrae\_fini (void)}\\
 Finalizes the tracing library and dumps the intermediate tracing buffers onto disk.\\
 {\bf NOTE:} This routine is called automatically by either {\tt MPI\_Finalize} or when the tracing is performed by the DynInst launcher.

 \item {\tt void Extrae\_event (unsigned event, unsigned value)}\\
 The Extrae\_event adds a single timestamped event into the tracefile. The event has two arguments: type and value.

 Some common use of events are:
  \begin{itemize}
   \item Identify loop iterations (or any code block): Given a loop, the user can set a unique type for the loop and a value related to the iterator value of the loop. For example:
    \begin{verbatim}
     for (i = 1; i <= MAX_ITERS; i++)
     {
       Extrae_event (1000, i);
       [original loop code]
     }
     Extrae_event (1000, 0);
    \end{verbatim}
   The last added call to Extrae\_event marks the end of the loop setting the event value to 0, which facilitates the analysis with Paraver.
   \item Identify user routines: Choosing a constant type (6000019 in this example) and different values for different routines (set to 0 to mark a "leave" event) 
    \begin{verbatim}
     void routine1 (void)
     {
      Extrae_event (6000019, 1);
      [routine 1 code]
      Extrae_event (6000019, 0);
     }

     void routine2 (void)
     {
      Extrae_event (6000019, 2);
      [routine 2 code]
      Extrae_event (6000019, 0);
     }
   \end{verbatim}
   \item Identify any point in the application using a unique combination of type and value.
  \end{itemize}

 \item {\tt void Extrae\_nevent (unsigned count, unsigned *types, unsigned *values)}\\
  Allows the user to place {\em count} events with the same timestamp at the given position.

 \item {\tt void Extrae\_counters (void)}\\
  Emits the value of the active hardware counters set. See chapter \ref{cha:XML} for further information.

 \item {\tt void Extrae\_eventandcounters (int event, int value)}\\
  This routine lets the user add an event and obtain the performance counters with one call and a single timestamp.

 \item {\tt void Extrae\_neventandcounters (unsigned count, unsigned *types, unsigned *values)}\\
  This routine lets the user add several events and obtain the performance counters with one call and a single timestamp.

 \item {\tt void Extrae\_shutdown (void)}\\
  Turns off the instrumentation.

 \item {\tt void Extrae\_restart (void)}\\
  Turns on the instrumentation.

 \item {\tt void Extrae\_previous\_hwc\_set (void)}\\
  Makes the previous hardware counter set defined in the XML file to be the active set (see section \ref{sec:XMLSectionMPI} for further information).

 \item {\tt void Extrae\_next\_hwc\_set (void)}\\
  Makes the following hardware counter set defined in the XML file to be the active set (see section \ref{sec:XMLSectionMPI} for further information).

 \item {\tt void Extrae\_set\_tracing\_tasks (int from, int to)}\\
  Allows the user to choose from which tasks (not {\em threads}!) store informartion in the tracefile

 \item {\tt void Extrae\_set\_options (int options)}\\
  Permits configuring several tracing options at runtime. The {\tt options} parameter has to be a bitwise or combination of the following options, depending on the user's needs:
  \begin{itemize}
   \item {\tt EXTRAE\_CALLER\_OPTION}\\
    Dumps caller information at each entry or exit point of the MPI routines. Caller levels need to be configured at XML (see chapter \ref{cha:XML}).
   \item {\tt EXTRAE\_HWC\_OPTION}\\
    Activates hardware counter gathering.
   \item {\tt EXTRAE\_MPI\_OPTION}\\
    Activates tracing of MPI calls.
   \item {\tt EXTRAE\_MPI\_HWC\_OPTION}\\
    Activates hardware counter gathering in MPI routines.
   \item {\tt EXTRAE\_OMP\_OPTION}\\
    Activates tracing of OpenMP runtime or outlined routines.
   \item {\tt EXTRAE\_OMP\_HWC\_OPTION}\\
    Activates hardware counter gathering in OpenMP runtime or outlined routines.
   \item {\tt EXTRAE\_UF\_HWC\_OPTION}\\
    Activates hardware counter gathering in the user functions.
  \end{itemize}

 \item {\tt void Extrae\_network\_counters (void)}\\
  Emits the value of the network counters if the system has this capability. {\em (Only available for systems with Myrinet GM/MX networks).}

 \item {\tt void Extrae\_network\_routes (int task)}\\
  Emits the network routes for an specific {\tt task}. {\em (Only available for systems with Myrinet GM/MX networks).}

 \item {\tt void Extrae\_user\_function (int enter)}\\
  Emits an event into the tracefile which references the source code (data includes: source line number, file name and function name). If {\tt enter} is 0 it marks an end (i.e., leaving the function), otherwise it marks the beginning of the routine. The user must be careful to place the call of this routine in places where the code is always executed, being careful not to place them inside {\tt if} and {\tt return} statements.
    \begin{verbatim}
     void routine1 (void)
     {
      Extrae_user_function (1);
      [routine 1 code]
      Extrae_user_function (0);
     }

     void routine2 (void)
     {
      Extrae_user_function (1);
      [routine 2 code]
      Extrae_user_function (0);
     }
   \end{verbatim}
   In order to gather performance counters during the execution of these calls, the {\tt user-functions} tag in the XML configuration and its {\tt counters} have to be both enabled.

\end{itemize}

\section{Extended API}\label{sec:ExtendedAPI}

{\em {\bf NOTE:} This API is in experimental stage and it is only available in C. Use it at your own risk!}

The extended API makes use of two special structures located in {\tt \$\{PREFIX\}/include/extrae\_types.h}. The structures are {\tt extrae\_UserCommunication} and {\tt extrae\_CombinedEvents}. The former is intended to encode an event that will be converted into a Paraver communication when its partner equivalent event has found. The latter is used to generate events containing multiple kinds of information at the same time.

\begin{verbatim}
struct extrae_UserCommunication
{
  extrae_user_communication_types_t type;
  extrae_comm_tag_t tag;
  unsigned size; /* size_t? */
  extrae_comm_partner_t partner;
  extrae_comm_id_t id;
};
\end{verbatim}

The structure {\tt extrae\_UserCommunication} contains the following fields:
\begin{itemize}
	\item {\tt type}\\
	Available options are:
	\begin{itemize}
		\item {\tt EXTRAE\_USER\_SEND}, if this event represents a send point.
		\item {\tt EXTRAE\_USER\_RECV}, if this event represents a receive point.
	\end{itemize}
	\item {\tt tag}\\
	The tag information in the communication record. 
	\item {\tt size}\\
	The size information in the communication record.
	\item {\tt partner}\\
	The partner of this communication (receive if this is a send or send if this is a receive). Partners (ranging from 0 to N-1) are considered across tasks whereas all threads share a single communication queue.
	\item {\tt id}\\
	An identifier that is used to match communications between partners.
\end{itemize}

\begin{verbatim}
struct extrae_CombinedEvents
{
  /* These are used as boolean values */
  int HardwareCounters;
  int Callers;
  int UserFunction;
  /* These are intended for N events */
  unsigned nEvents;
  extrae_type_t  *Types;
  extrae_value_t *Values;
  /* These are intended for user communication records */
  unsigned nCommunications;
  extrae_user_communication_t *Communications;
};
\end{verbatim}

The structure {\tt extrae\_CombinedEvents} contains the following fields:
\begin{itemize}
	\item {\tt HardwareCounters}\\
	Set to non-zero if this event has to gather hardware performance counters.
	\item {\tt Callers}\\
	Set to non-zero if this event has to emit callstack information.
	\item {\tt UserFunction}\\
	Available options are:
	\begin{itemize}
		\item {\tt EXTRAE\_USER\_FUNCTION\_NONE}, if this event should not provide information about user routines.
		\item {\tt EXTRAE\_USER\_FUNCTION\_ENTER}, if this event represents the starting point of a user routine.
		\item {\tt EXTRAE\_USER\_FUNCTION\_LEAVE}, if this event represents the ending point of a user routine.
	\end{itemize}
	\item {\tt nEvents}\\
	Set the number of events given in the {\tt Types} and {\tt Values} fields.
	\item {\tt Types}\\
	A pointer containing {\tt nEvents} type that will be stored in the trace.
	\item {\tt Values}\\
	A pointer containing {\tt nEvents} values that will be stored in the trace.
	\item {\tt nCommunications}\\
	Set the number of communications given in the {\tt Communications} field.
	\item {\tt Communications}\\
	A pointer to {\tt extrae\_UserCommunication} structures containing {\tt nCommunications} elements that represent the involved communications.
\end{itemize}

The extended API contains the following routines:
\begin{itemize}
	\item {\tt void Extrae\_init\_UserCommunication (struct extrae\_UserCommunication *)}\\
	Use this routine to initialize an extrae\_UserCommunication structure.
	\item {\tt void Extrae\_init\_CombinedEvents (struct extrae\_CombinedEvents *)}\\
	Use this routine to initialize an extrae\_CombinedEvents structure.
	\item {\tt void Extrae\_emit\_CombinedEvents (struct extrae\_CombinedEvents *)}\\
	Use this routine to emit to the tracefile the events set in the extrae\_CombinedEvents given.
	\item {\tt void Extrae\_resume\_virtual\_thread (unsigned vthread)}\\
	This routine changes the thread identifier so as to be vthread in the final tracefile. {\em Improper use of this routine may result in corrupt tracefiles.}
	\item {\tt void Extrae\_suspend\_virtual\_thread (void)}\\
	This routine recovers the original thread identifier (given by routines like pthread\_self or omp\_get\_thread\_num, for instance).
	\item {\tt void Extrae\_register\_stacked\_type (extrae\_type\_t type)}\\
	Registers which event types are required to be managed in a stack way whenever {\tt void Extrae\_resume\_virtual\_thread} or {\tt void Extrae\_suspend\_virtual\_thread} are called.
	\item {\tt void Extrae\_set\_threadid\_function (unsigned (*threadid\_function)(void))}\\
	Defines the routine that will be used as a thread identifier inside the tracing facility.
	\item {\tt void Extrae\_set\_numthreads\_function (unsigned (*numthreads\_function)(void))}\\
	Defines the routine that will count all the executing threads inside the tracing facility.
	\item {\tt void Extrae\_set\_taskid\_function (unsigned (*taskid\_function)(void))}\\
	Defines the routine that will be used as a task identifier inside the tracing facility.
	\item {\tt void Extrae\_set\_numtasks\_function (unsigned (*numtasks\_function)(void))}\\
	Defines the routine that will count all the executing tasks inside the tracing facility.
	\item {\tt void Extrae\_set\_barrier\_tasks\_function (void (*barriertasks\_function)(void))}\\
	Establishes the barrier routine among tasks. It is needed for synchronization purposes.
\end{itemize}

\section{Special considerations for Cell Broadband Engine tracing package}

Instead of including {\tt \$\{EXTRAE\_HOME\}/include/extrae\_user\_events.h} include:
\begin{itemize}
 \item {\tt \$\{EXTRAE\_HOME\}/include/ppu\_trace\_sdk2.h} on the PPE side, and,
 \item {\tt \$\{EXTRAE\_HOME\}/include/sputrace\_user\_events.h} on the SPE side.
\end{itemize}

\subsection{PPE side}\label{subsec:PPEside}

The routines shown on section \ref{sec:BasicAPI} are available for the PPE element. In addition, two additional routines are available to control the creation and finalization of the SPE threads. These routines are:

\begin{itemize}

 \item {\tt int  CELLtrace\_init (int spus, spe\_context\_ptr\_t * spe\_ids)}\\
 Contacts with the SPE thread to initialize once the SPE tracing environment. Such call has to be synchronized with the invocation of {\tt SPUtrace\_init} (see \ref{subsec:SPEside}) call on the SPE side due to the presence of message passing using the mailboxes. The routine receives the total number of contexts created by the Cell SDK {\tt spe\_context\_create} and a vector pointing to those contexts. Each of those contexts will reference to a single SPE thread created by a call to {\tt pthread\_create}.

 \item {\tt void CELLtrace\_fini (void)}\\
 Waits for the finalization of all the threads registered in {\tt CELLtrace\_init} and dumps their intermediate tracing buffers.

\end{itemize}

\subsection{SPE side}\label{subsec:SPEside}

Due to the lack of parallel paradigms and hardware counters inside the SPE element, the SPE tracing library is a subset of the typical tracing library. The following API calls are available for the SPE element:

\begin{itemize}

 \item {\tt void SPUtrace\_init (void)}\\
 Initializes the tracing package in the SPE side. It has to be synchronized with {\tt CELLtrace\_init} due to the message passing using mailboxes.

 \item {\tt void SPUtrace\_fini (void)}\\
 Notifies the finalization of the work performed in the SPE thread and transfers the tracing buffer to the PPE element.

 \item {\tt void SPUtrace\_event (unsigned event, unsigned value)}\\
 Has the same semantics as {\tt Extrae\_event}.

 \item {\tt void SPUtrace\_nevent (unsigned count, unsigned *types, unsigned *values)}\\
 Has the same semantics as {\tt Extrae\_nevent}.

\end{itemize}




\chapter{Merging process}\label{cha:Merging}

Once the application has finished, and if the automatic merge process is not setup, the merge must be executed manually. Here we detail how to run the merge process manually.

The inserted probes in the instrumented binary are responsible for gathering performance metrics of each task/thread and for each of them several files are created where the XML configuration file specified (see section \ref{sec:XMLSectionStorage}). Such files are:

\begin{itemize}
 \item As many {\tt .mpit} files as tasks and threads where running the target application. Each file contains information gathered by the specified task/thread in raw binary format.
 \item A single {\tt .mpits} file that contain a list of related {\tt .mpit} files.
 \item If the DynInst based instrumentation package was used, an addition {\tt .sym} file that contains some symbolic information gathered by the DynInst library.
\end{itemize}

In order to use Paraver, those intermediate files ({\it i.e.}, {\tt .mpit} files) must be merged and translated into Paraver trace file format. The same applies if the user wants to use the Dimemas simulator. To proceed with any of these translation all the intermediate trace files must be merged into a single trace file using one of the available mergers in the {\tt bin} directory (see table \ref{tab:MergerDescription}).

The target trace type is defined in the XML configuration file used at the instrumentation step (see section \ref{sec:XMLSectionTraceConfiguration}), and it has match with the merger used (mpi2prv and mpimpi2prv for Paraver and mpi2dim and mpimpi2dim for Dimemas). However, it is possible to force the format nevertheless the selection done in the XML file using the parameters {\tt -paraver} or {\tt -dimemas}\footnote{The timing mechanism differ in Paraver/Dimemas at the instrumentation level. If the output trace format does not correspond with that selected in the XML some timing inaccuracies may be present in the final tracefile. Such inaccuracies are known to be higher due to clock granularity if the XML is set to obtain Dimemas tracefiles but the resulting tracefile is forced to be in Paraver format.}.

\begin{table}[ht]
\centerline {
\begin{tabular}{| l | l |}
\hline
{\bf Binary} & {\bf Description} \\
\hline
mpi2prv      & Sequential version of the Paraver merger.\\
\hline
mpi2dim      & Sequential version of the Dimemas merger.\\
\hline
mpimpi2prv   & Parallel version of the Paraver merger.\\
\hline
mpimpi2dim   & Parallel version of the Dimemas merger.\\
\hline
\end{tabular}
}
\caption{Description of the available mergers in the \TRACE package.}
\label{tab:MergerDescription}
\end{table}

\section{Paraver merger}

As stated before, there are two Paraver mergers: {\tt mpi2prv} and {\tt mpimpi2prv}. The former is for use in a single processor mode while the latter is meant to be used with multiple processors using MPI (and cannot be run using one MPI task).

Paraver merger receives a set of intermediate trace files and generates three files with the same name (which is set with the {\tt -o} option) but differ in the extension. The Paraver trace itself (.prv file) that contains timestamped records that represent the  information gathered during the execution of the instrumented application. It also generates the Paraver Configuration File (.pcf file), which is responsible for translating values contained in the Paraver trace into a more human readable values. Finally, it also generates a file containing the distribution of the application across the cluster computation resources (.row file).

The following sections describe the available options for the Paraver mergers. Typically, options available for single processor mode are also available in the parallel version, unless specified.

\subsection{Sequential Paraver merger}

These are the available options for the sequential Paraver merger:

\begin{itemize}
 \item {\tt -d} or {\tt -dump}\\
 Dumps the information stored in the intermediate trace files.
 \item {\tt -dump-without-time}\\
 The information dumped with {\tt -d} (or {\tt -dump}) does not show the timestamp.
 \item {\tt -e BINARY}\\
 Uses the given BINARY to translate addresses that are stored in the intermediate trace files into useful information (including function name, source file and line). The application has to be compiled with {\tt -g} flag so as to obtain valuable information.
 \item {\tt -evtnum N}\\
 Partially processes (up to N events) the intermediate trace files to generate the Dimemas tracefile.
 \item {\tt -f FILE.mpits} {\em (where {\tt FILE.mpits} file is generated by the instrumentation)}\\
 The merger uses the given file (which contains a list of intermediate trace files of a single executions) instead of giving set of intermediate trace files.\\
 This option looks first for each file listed in the parameter file. Each contained file is searched in the absolute given path, if it does not exist, then it's searched in the current directory.
 \item {\tt -f-relative FILE.mpits} {\em (where {\tt FILE.mpits} file is generated by the instrumentation)}\\
 This options behaves like the -f options but looks for the intermediate files in the current directory.
 \item {\tt -f-absolute FILE.mpits} {\em (where {\tt FILE.mpits} file is generated by the instrumentation)}\\
 This options behaves like the -f options but uses the full path of every intermediate file so as to locate them.
 \item {\tt -h}\\
 Provides minimal help about merger options.
 \item {\tt -keep-mpits} (or inversely, {\tt -no-keep-mpits})\\
 Tells the merger to keep (or remove) the intermediate tracefiles after the trace generation.
 \item {\tt -maxmem M}\\
 The last step of the merging process will be limited to use {\em M} megabytes of memory. By default, M is 512.
 \item {\tt -s FILE.sym} {\em (where {\tt FILE.sym} file is generated with the DynInst instrumentator)}\\
 Passes information regarding instrumented symbols into the merger to aid the Paraver analysis. If {\tt -f}, {\tt -f-relative} or {\tt -f-absolute} paramters are given, the merge process will try to automatically load the symbol file associated to that FILE.mpits file.
 \item {\tt -no-syn}\\
 If set, the merger will not attempt to synchronize the different tasks. This is useful when merging intermediate files obtained from a single node (and thus, share a single clock).
 \item {\tt -o FILE.prv}\\
 Choose the name of the target Paraver tracefile.
 \item {\tt -o FILE.prv.gz}\\
 Choose the name of the target Paraver tracefile compressed using the libz library.
 \item {\tt -remove-files}\\
 The merging process removes the intermediate tracefiles when succesfully generating the Paraver tracefile.
 \item {\tt -skip-sendrecv}\\
 Do not match point to point communications issued by {\tt MPI\_Sendrecv} or {\tt MPI\_Sendrecv\_replace}.
 \item {\tt -sort-addresses}\\
 Sort event values that reference source code locations so as the values are sorted by file name first and then line number (enabled by default).
 \item {\tt -split-states}\\
 Do not join consecutive states that are the same into a single one.
 \item {\tt -syn}\\
 If different nodes are used in the execution of a tracing run, there can exist some clock differences on all the nodes. This option makes mpi2prv to recalculate all the timings based on the end of the MPI\_Init call. This will usually lead to "synchronized" tasks, but it will depend on how the clocks advance in time.
 \item {\tt -syn-node}\\
 If different nodes are used in the execution of a tracing run, there can exist some clock differences on all the nodes. This option makes mpi2prv to recalculate all the timings based on the end of the MPI\_Init call and the node where they ran. This will usually lead to better synchronized tasks than using -syn, but, again, it will depend on how the clocks advance in time.
 \item {\tt -trace-overwrite} (or inversely, {\tt -no-trace-overwrite})\\
 Tells the merger to overwrite (or not) the final tracefile if it already exists. If the tracefile exists and {\tt -no-trace-overwrite} is given, the tracefile name will have an increasing numbering in addition to the name given by the user.
 \item {\tt -unique-caller-id}\\
 Choose whether use a unique value identifier for different callers locations (MPI calling routines, user routines, OpenMP outlined routines andpthread routines).
\end{itemize}

\subsection{Parallel Paraver merger}

These options are specific to the parallel version of the Paraver merger:

\begin{itemize}
 \item {\tt -block}\\
  Intermediate trace files will be distributed in a block fashion instead of a cyclic fashion to the merger.
 \item {\tt -cyclic}\\
	Intermediate trace files will be distributed in a cyclic fashion instead of a block fashion to the merger.
 \item {\tt -size}\\
	The intermediate trace files will be sorted by size and then assigned to processors in a such manner that each processor receives approximately the same size.
 \item {\tt -consecutive-size}\\
	Intermediate trace files will be distributed consecutively to processors but trying to distribute the overall size equally among processors.
 \item {\tt -use-disk-for-comms}\\
 Use this option if your memory resources are limited. This option uses an alternative matching communication algorithm that saves memory but uses intensively the disk.
 \item{\tt -tree-fan-out N}\\
  Use this option to instruct the merger to generate the tracefile using a tree-based topology. This should improve the performance when using a large number of processes at the merge step. Depending on the combination of processes and the width of the tree, the merger will need to run several stages to generate the final tracefile.\\
  The number of processes used in the merge process must be equal or greater than the {\em N} parameter. If it is not, the merger itself will automatically set the width of the tree to the number of processes used.
\end{itemize}

\section{Dimemas merger}

As stated before, there are two Dimemas mergers: {\tt mpi2dim} and {\tt mpimpi2dim}. The former is for use in a single processor mode while the latter is meant to be used with multiple processors using MPI.

In contrast with Paraver merger, Dimemas mergers generate a single output file with the .dim extension that is suitable for the Dimemas simulator from the given intermediate trace files..

These are the available options for both Dimemas mergers:

\begin{itemize}
 \item {\tt -evtnum N}\\
 Partially processes (up to N events) the intermediate trace files to generate the Dimemas tracefile.
 \item {\tt -f FILE.mpits} {\em (where {\tt FILE.mpits} file is generated by the instrumentation)}\\
 The merger uses the given file (which contains a list of intermediate trace files of a single executions) instead of giving set of intermediate trace files.\\
 This option takes only the file name of every intermediate file so as to locate them.
 \item {\tt -f-relative FILE.mpits} {\em (where {\tt FILE.mpits} file is generated by the instrumentation)}\\
 This options works exactly as the -f option.
 \item {\tt -f-absolute FILE.mpits} {\em (where {\tt FILE.mpits} file is generated by the instrumentation)}\\
 This options behaves like the -f options but uses the full path of every intermediate file so as to locate them.
 \item {\tt -h}\\
 Provides minimal help about merger options.
 \item {\tt -maxmem M}\\
 The last step of the merging process will be limited to use {\em M} megabytes of memory. By default, M is 512.
 \item {\tt -o FILE.dim}\\
 Choose the name of the target Dimemas tracefile.
\end{itemize}

\section{Environment variables}

There are some environment variables that are related Two environment variables 

\subsection{Environment variables suitable to Paraver merger}

\subsubsection{EXTRAE\_LABELS}

This environment variable lets the user add custom information to the generated Paraver Configuration File ({\tt .pcf}). Just set this variable to point to a file containing labels for the unknown (user) events.

The format for the file is:

\begin{verbatim}
  EVENT_TYPE
  0 [type1] [label1]
  0 [type2] [label2]
  ...
  0 [typeK] [labelK]
\end{verbatim}

Where {\tt [typeN]} is the event value and {\tt [labelN]} is the description for the event with value [typeN].
It is also possible to link both type and value of an event:

\begin{verbatim}
  EVENT_TYPE
  0 [type] [label]
  VALUES
  [value1] [label1]
  [value2] [label2]
  ...
  [valueN] [labelN]
\end{verbatim}

With this information, Paraver can deal with both type and  value when giving textual information to the end user. If Paraver does not find any information for an event/type it will shown it in numerical form.

\subsubsection{MPI2PRV\_TMP\_DIR}

Points to a directory where all intermediate temporary files will be stored. These files will be removed as soon the application ends.

\subsection{Environment variables suitable to Dimemas merger}

\subsubsection{MPI2DIM\_TMP\_DIR}

Points to a directory where all intermediate temporary files will be stored. These files will be removed as soon the application ends.



\chapter{Examples}\label{cha:Examples}

We present here three different examples of generating a Paraver tracefile. First example requires the package to be compiled with DynInst libraries. Second example uses the {\tt LD\_PRELOAD} mechanism to interpose code in the application. Such mechanism is available in Linux and FreeBSD operating systems and only works when the application uses dynamic libraries. Finally, there is an example using the static library of the instrumentation package.

\section{DynInst based examples}\label{sec:Examples_DynInst}

\subsection{Generating the intermediate files}\label{subsec:Examples_DynInst_Intermediate}

\section{LD\_PRELOAD based examples}\label{sec:Examples_LDPRELOAD}

LD\_PRELOAD interposition mechanism only works for binaries that are linked against shared libraries. This interposition is done by the runtime loader by substituting the original symbols by those provided by the instrumentation package. This mechanism is known to work on Linux and FreeBSD operating systems, although it may be available on other operating systems (even using different names\footnote{Look at \url{http://www.fortran-2000.com/ArnaudRecipes/sharedlib.html} for further information.}) they are not tested.

\subsection{Generating the intermediate files}\label{subsec:Examples_LDPRELOAD_Intermediate}

The following script preloads the libmpitrace library to instrument MPI calls of the application passed as an argument (tune {\tt EXTRAE\_HOME} according to your installation).

\begin{Verbatim}[frame=single,numbers=left,labelposition=topline,label=trace.sh]
#!/bin/sh

export EXTRAE_HOME=WRITE-HERE-THE-PACKAGE-LOCATION
export EXTRAE_CONFIG_FILE=extrae.xml
export LD_PRELOAD=${EXTRAE_HOME}/lib/libmpitrace.so

## Run the desired program
$*
\end{Verbatim}

The previous script can be found in the share/example/MPI/ld-preload directory in your tracing package directory. Copy the script to one of your directories, tune the {\tt EXTRAE\_HOME} environment variable and make the script executable (using {\tt chmod u+x}). Also copy the XML configuration extrae.xml file from the share/example/MPI directory instrumentation package to the current directory. This file is used to configure the whole behavior of the instrumentation package (there is more information about the XML file on chapter \ref{cha:XML}). The last line in the script, $\$\ast$, executes the arguments given to the script, so as you can run the instrumentation by simply adding the script in between your execution command.

Regarding the execution, if you run MPI applications from the command-line, you can issue the typical mpirun command as:

\graybox{\tt \$\{MPI\_HOME\}/bin/mpirun -np N ./trace.sh mpi-app}

where, {\tt \$\{MPI\_HOME\}} is the directory for your MPI installation, {\tt N} is the number of MPI tasks you want to run and {\tt mpi-app} is the binary of the MPI application you want to run.

However, if you execute your MPI applications through a queue system you may need to write a submission script. The following script is an example of a submission script for MOAB/Slurm queuing system using the aforementioned trace.sh script for an execution of the {\tt mpi-app} on two processors.

\begin{Verbatim}[frame=single,numbers=left,labelposition=topline,label=slurm-trace.sh]
#! /bin/bash
#@ job_name         = trace_run
#@ output           = trace_run%j.out
#@ error            = trace_run%j.out
#@ initialdir       = .
#@ class            = bsc_cs
#@ total_tasks      = 2
#@ wall_clock_limit = 00:30:00

srun ./trace.sh mpi_app 
\end{Verbatim}

If your system uses LoadLeveler your job script may look like:

\begin{Verbatim}[frame=single,numbers=left,labelposition=topline,label=ll.sh]
#! /bin/bash
#@ job_type = parallel
#@ output = trace_run.ouput
#@ error = trace_run.error
#@ blocking = unlimited
#@ total_tasks = 2
#@ class = debug
#@ wall_clock_limit = 00:10:00
#@ restart = no
#@ group = bsc41 
#@ queue

export MLIST=/tmp/machine_list ${$}
/opt/ibmll/LoadL/full/bin/ll_get_machine_list > ${MLIST}
set NP = `cat ${MLIST} | wc -l`

${MPI_HOME}/mpirun -np ${NP} -machinefile ${MLIST} ./trace.sh ./mpi-app

rm ${MLIST}
\end{Verbatim}

Besides the job specification given in lines 1-11, there are commands of particular interest. Lines 13-15 are used to know which and how many nodes are involved in the computation. Such information information is given to the {\tt mpirun} command to proceed with the execution. Once the execution finished, the temporal file created on line 14 is removed on line 19.

\section{Statically linked based examples}\label{sec:Examples_static}

This is the basic instrumentation method suited for those installations that neither support DynInst nor LD\_PRELOAD, or require adding some manual calls to the \TRACE API.

\subsection{Linking the application}\label{subsec:Examples_static_link}

To get the instrumentation working on your code, first you have to link your application with the \TRACE libraries. There are installed examples in your package distribution under share/examples directory. There you can find MPI, OpenMP, pthread and sequential examples depending on the support at configure time.

Consider the example Makefile found in share/examples/MPI/static:

\begin{Verbatim}[frame=single,numbers=left,labelposition=topline,label=Makefile]
MPI_HOME = /gpfs/apps/MPICH2/mx/1.0.7..2/64
EXTRAE_HOME = /home/bsc41/bsc41273/foreign-pkgs/extrae-11oct-mpich2/64
PAPI_HOME = /gpfs/apps/PAPI/3.6.2-970mp-patched/64
XML2_LDFLAGS = -L/usr/lib64
XML2_LIBS = -lxml2

F77 = $(MPI_HOME)/bin/mpif77 
FFLAGS = -O2
FLIBS = $(EXTRAE_HOME)/lib/libmpitracef.a \
        -L$(PAPI_HOME)/lib -lpapi -lperfctr \
        $(XML2_LDFLAGS) $(XML2_LIBS)

all: mpi_ping

mpi_ping: mpi_ping.f
	$(F77) $(FFLAGS) mpi_ping.f $(FLIBS) -o mpi_ping

clean:
	rm -f mpi_ping *.o pingtmp? TRACE.*
\end{Verbatim}

Lines 2-5 are definitions of some Makefile variables to set up the location of different packages needed by the instrumentation. In particular, {\tt EXTRAE\_HOME} sets where the \TRACE package directory is located. In order to link your application with \TRACE you have to add its libraries in the link stage (see lines 9-11 and 16). Besides {\tt libmpitracef.a} we also add some PAPI libraries ({\tt -lpapi}, and its dependency (which you may or not need {\tt -lperfctr}), and the libxml2 parsing library ({\tt -lxml2}).

\subsection{Generating the intermediate files}\label{subsec:Examples_static_Intermediate}

Executing an application with the statically linked version of the instrumentation package is very similar as the method shown in subsection \ref{subsec:Examples_LDPRELOAD_Intermediate}. There is, however, a difference: do not set LD\_PRELOAD in {\tt trace.sh}.

\begin{Verbatim}[frame=single,numbers=left,labelposition=topline,label=trace.sh]
#!/bin/sh

export EXTRAE_HOME=WRITE-HERE-THE-PACKAGE-LOCATION
export EXTRAE_CONFIG_FILE=extrae.xml
export LD_LIBRARY_PATH=${EXTRAE_HOME}/lib:\
                       /gpfs/apps/MPICH2/mx/1.0.7..2/64/lib:\
                       /gpfs/apps/PAPI/3.6.2-970mp-patched/64/lib

## Run the desired program
$*
\end{Verbatim}

See section \ref{subsec:Examples_LDPRELOAD_Intermediate} to know how to run this script either through command line or queue systems.

\section{Generating the final tracefile}\label{subsec:Examples_LDPRELOAD_Final}

Independently from the tracing method chosen, it is necessary to translate the intermediate tracefiles into a Paraver tracefile. The Paraver tracefile can be generated automatically (if the tracing package and the XML configuration file were set up accordingly, see chapters \ref{cha:Configuration} and \ref{cha:XML}) or manually. In case of using the automatic merging process, it will use all the resources allocated for the application to perform the merge once the application ends.

To manually generate the final Paraver tracefile issue the following command:

\graybox{\tt \$\{EXTRAE\_HOME\}/bin/mpi2prv -f TRACE.mpits -e mpi-app -o trace.prv}

This command will convert the intermediate files generated in the previous step into a single Paraver tracefile. The {\tt TRACE.mpits} is a file generated automatically by the instrumentation and contains a reference to all the intermediate files generated during the execution run. The {\tt -e} parameter receives the application binary {\tt mpi-app} in order to perform translations from addresses to source code. To use this feature, the binary must have been compiled with debugging information. Finally, the {\tt -o} flag tells the merger how the Paraver tracefile will be named (trace.prv in this case).



\appendix

\chapter{An example of \TRACE XML configuration file}\label{cha:wholeXML}

\begin{verbatim}
<?xml version='1.0'?>

<trace enabled="yes"
 home="@sed_MYPREFIXDIR@"
 initial-mode="detail"
 type="paraver"
 xml-parser-id="@sed_XMLID@"
>
  <mpi enabled="yes">
    <counters enabled="yes" />
  </mpi>

  <pacx enabled="no">
    <counters enabled="yes" />
  </pacx>

  <pthread enabled="yes">
    <locks enabled="no" />
    <counters enabled="yes" />
  </pthread>

  <openmp enabled="yes">
    <locks enabled="no" />
    <counters enabled="yes" />
  </openmp>

  <callers enabled="yes">
    <mpi enabled="yes">1-3</mpi>
    <pacx enabled="no">1-3</pacx>
    <sampling enabled="no">1-5</sampling>
  </callers>

  <user-functions enabled="no"
    list="/home/bsc41/bsc41273/user-functions.dat"
    exclude-automatic-functions="no">
    <counters enabled="yes" />
  </user-functions>

  <counters enabled="yes">
    <cpu enabled="yes" starting-set-distribution="1">
      <set enabled="yes" domain="all" changeat-globalops="5">
        PAPI_TOT_INS,PAPI_TOT_CYC,PAPI_L1_DCM
        <sampling enabled="no" period="100000000">PAPI_TOT_CYC</sampling>
      </set>
      <set enabled="yes" domain="user" changeat-globalops="5">
        PAPI_TOT_INS,PAPI_FP_INS,PAPI_TOT_CYC
      </set>
    </cpu>
    <network enabled="yes" />
    <resource-usage enabled="yes" />
  </counters>

  <storage enabled="no">
    <trace-prefix enabled="yes">TRACE</trace-prefix>
    <size enabled="no">5</size>
    <temporal-directory enabled="yes">/scratch</temporal-directory>
    <final-directory enabled="yes">/gpfs/scratch/bsc41/bsc41273</final-directory>
    <gather-mpits enabled="no" />
  </storage>

  <buffer enabled="yes">
    <size enabled="yes">150000</size>
    <circular enabled="no" />
  </buffer>

  <trace-control enabled="yes">
    <file enabled="no" frequency="5M">/gpfs/scratch/bsc41/bsc41273/control</file>
    <global-ops enabled="no">10</global-ops>
    <remote-control enabled="yes">
      <mrnet enabled="yes" target="150" analysis="spectral" start-after="30">
        <clustering max_tasks="26" max_points="8000"/>
        <spectral min_seen="1" max_periods="0" num_iters="3" signals="DurBurst,InMPI"/>
      </mrnet>
      <signal enabled="no" which="USR1"/>
    </remote-control>
  </trace-control> 

  <others enabled="yes">
    <minimum-time enabled="no">10m</minimum-time>
  </others>

  <bursts enabled="no">
    <threshold enabled="yes">500u</threshold>
    <mpi-statistics enabled="yes" />
    <pacx-statistics enabled="no" />
  </bursts>

  <cell enabled="no">
    <spu-file-size enabled="yes">5</spu-file-size>
    <spu-buffer-size enabled="yes">64</spu-buffer-size>
    <spu-dma-channel enabled="yes">2</spu-dma-channel>
  </cell>

  <sampling enabled="no" type="default" period="50m" variability="10m"/>

	<opencl enabled="no" />

	<cuda enabled="no" />

  <merge enabled="yes" 
    synchronization="default"
    binary="mpi_ping"
    tree-fan-out="16"
    max-memory="512"
    joint-states="yes"
    keep-mpits="yes"
    sort-addresses="yes"
    overwrite="yes"
  >
    mpi_ping.prv 
  </merge>

</trace>
\end{verbatim}




\chapter{Environment variables}\label{cha:EnvVar}

Although \TRACE is configured through an XML file (which is pointed by the {\tt EXTRAE\_CONFIG\_FILE}), it also supports minimal configuration to be done via environment variables for those systems that do not have the library responsible for parsing the XML files ({\em i.e.,} libxml2).

This appendix presents the environment variables the \TRACE package uses if {\tt EXTRAE\_CONFIG\_FILE} is not set and a description. For those environment variable that refer to XML 'enabled' attributes ({\em i.e.}, that can be set to "yes" or "no") are considered to be enabled if their value are defined to 1.
 
\begin{landscape}
\begin{table}
\centerline {
\begin{tabular}{p{7cm} p{14cm}}
  \hline
  {\bf Environment variable} & {\bf Description}\\
  \hline
  \cellcolor{tabbg2} & \cellcolor{tabbg2} Set the number of records that the instrumentation buffer can\\
  \cellcolor{tabbg2}\multirow{-2}{*}{EXTRAE\_BUFFER\_SIZE} & \cellcolor{tabbg2} hold before flushing them.\\
  \cellcolor{tabbg1} & \cellcolor{tabbg1}See section \ref{subsec:ProcessorPerformanceCounters}. Just one set can be defined. Counters (in PAPI)\\
  \cellcolor{tabbg1}\multirow{-2}{*}{EXTRAE\_COUNTERS} & \cellcolor{tabbg1}groups (in PMAPI) are given separated by commas.\\
  \cellcolor{tabbg2}EXTRAE\_CONTROL\_FILE & \cellcolor{tabbg2}The instrumentation will be enabled only when the file pointed exists.\\
  \cellcolor{tabbg1} & \cellcolor{tabbg1}Starts the instrumentation when the specified number of global collectives\\ 
  \cellcolor{tabbg1}\multirow{-2}{*}{EXTRAE\_CONTROL\_GLOPS} & \cellcolor{tabbg1}have been executed.\\
  \cellcolor{tabbg2}EXTRAE\_CONTROL\_TIME & \cellcolor{tabbg2}Checks the file pointed by {\tt EXTRAE\_CONTROL\_FILE} at this period.\\
  \cellcolor{tabbg1} & \cellcolor{tabbg1}Specifies where temporal files will be created during\\
  \cellcolor{tabbg1}\multirow{-2}{*}{EXTRAE\_DIR} & \cellcolor{tabbg1}instrumentation.\\
  \cellcolor{tabbg2}EXTRAE\_DISABLE\_MPI & \cellcolor{tabbg2}Disable MPI instrumentation.\\
  \cellcolor{tabbg1}EXTRAE\_DISABLE\_OMP & \cellcolor{tabbg1}Disable OpenMP instrumentation.\\
  \cellcolor{tabbg2}EXTRAE\_DISABLE\_PTHREAD & \cellcolor{tabbg2}Disable pthread instrumentation.\\
  \cellcolor{tabbg1}EXTRAE\_DISABLE\_PACX & \cellcolor{tabbg1}Disable PACX instrumentation.\\
  \cellcolor{tabbg2}EXTRAE\_FILE\_SIZE & \cellcolor{tabbg2}Set the maximum size (in Mbytes) for the intermediate trace file.\\
  \cellcolor{tabbg1}                  & \cellcolor{tabbg1}List of routine to be instrumented, as described in \ref{sec:XMLSectionUF} using the\\
  \cellcolor{tabbg1}EXTRAE\_FUNCTIONS & \cellcolor{tabbg1}GNU C {\tt -finstrument-functions} or the IBM XL {\tt -qdebug=function\_trace}\\
  \cellcolor{tabbg1}                  & \cellcolor{tabbg1}option at compile and link time. \\
  \cellcolor{tabbg2} & \cellcolor{tabbg2}Specify if the performance counters should be collected when a\\
  \cellcolor{tabbg2}\multirow{-2}{*}{EXTRAE\_FUNCTIONS\_COUNTERS\_ON} & \cellcolor{tabbg2}user function event is emitted.\\
  \cellcolor{tabbg1}EXTRAE\_FINAL\_DIR & \cellcolor{tabbg1}Specifies where files will be stored when the application ends.\\
  \cellcolor{tabbg2} & \cellcolor{tabbg2}Gather intermediate trace files into a single directory\\
  \cellcolor{tabbg2}\multirow{-2}{*}{EXTRAE\_GATHER\_MPITS} & (\em{this is only available when instrumenting MPI applications}).\\
  \cellcolor{tabbg1}EXTRAE\_HOME & \cellcolor{tabbg1}Points where the \TRACE is installed.\\
  \cellcolor{tabbg2} & \cellcolor{tabbg2}Choose whether the instrumentation runs in in {\tt detail} or\\
  \cellcolor{tabbg2}\multirow{-2}{*}{EXTRAE\_INITIAL\_MODE} & \cellcolor{tabbg2}in {\tt bursts} mode.\\
  \cellcolor{tabbg1}EXTRAE\_BURST\_THRESHOLD & \cellcolor{tabbg1}Specify the threshold time to filter running bursts.\\
  \cellcolor{tabbg2}EXTRAE\_MINIMUM\_TIME & \cellcolor{tabbg2}Specify the minimum amount of instrumentation time.\\
  \hline
\end{tabular}
}
\caption{Set of environment variables available to configure \TRACE}
\label{tab:EnvironmentVariables}
\end{table}

\end{landscape}

\begin{landscape}

\begin{table}
\centerline {
\begin{tabular}{p{7cm} p{14cm}}
  \hline
  {\bf Environment variable} & {\bf Description}\\
	\hline
  \cellcolor{tabbg2} EXTRAE\_MPI\_CALLER  & \cellcolor{tabbg2} Choose which MPI calling routines should be dumped into the tracefile.\\
  \cellcolor{tabbg1} EXTRAE\_MPI\_COUNTERS\_ON & \cellcolor{tabbg1} Set to 1 if MPI must report performace counter values.\\
  \cellcolor{tabbg2} & \cellcolor{tabbg2} Set to 1 if basic MPI statistics must be collected in burst mode\\
  \cellcolor{tabbg2} \multirow{-2}{*}{EXTRAE\_MPI\_STATISTICS} & \cellcolor{tabbg2} {\em (Only available in systems with Myrinet GM/MX networks)}.\\
  \cellcolor{tabbg1} EXTRAE\_NETWORK\_COUNTERS & \cellcolor{tabbg1}  Set to 1 to dump network performance counters at flush points.\\
  \cellcolor{tabbg2} EXTRAE\_PTHREAD\_COUNTERS\_ON & \cellcolor{tabbg2} Set to 1 if pthread must report performance counters values. \\
  \cellcolor{tabbg1} EXTRAE\_OMP\_COUNTERS\_ON & \cellcolor{tabbg1} Set to 1 if OpenMP must report performance counters values. \\
  \cellcolor{tabbg2} EXTRAE\_PTHREAD\_LOCKS & \cellcolor{tabbg2} Set to 1 if pthread locks have to be instrumented.\\
  \cellcolor{tabbg1} EXTRAE\_OMP\_LOCKS & \cellcolor{tabbg1} Set to 1 if OpenMP locks have to be instrumented.\\
  \cellcolor{tabbg2} EXTRAE\_ON & \cellcolor{tabbg2} Enables instrumentation\\
  \cellcolor{tabbg1} EXTRAE\_PACX\_CALLER & \cellcolor{tabbg1} Choose which PACX calling routines should be dumped into the tracefile.\\
  \cellcolor{tabbg2} EXTRAE\_PACX\_COUNTERS\_ON & \cellcolor{tabbg2} Set to 1 if PACX must report performace counter values.\\
  \cellcolor{tabbg1} EXTRAE\_PACX\_STATISTICS & \cellcolor{tabbg1} Set to 1 if basic PACX statistics must be collected in burst mode.\\
  \cellcolor{tabbg2} EXTRAE\_PROGRAM\_NAME & \cellcolor{tabbg2} Specify the prefix of the resulting intermediate trace files.\\ 
  \cellcolor{tabbg1} EXTRAE\_SAMPLING\_CALLER & \cellcolor{tabbg1} Determines the callstack segment stored through time-sampling capabilities.\\
  \cellcolor{tabbg2} & \cellcolor{tabbg2} Determines domain for sampling clock.\\
  \cellcolor{tabbg2} \multirow{-2}{*}{EXTRAE\_SAMPLING\_CLOCKTYPE} & \cellcolor{tabbg2} Options are: DEFAULT, REAL, VIRTUAL and PROF.\\
  \cellcolor{tabbg1} EXTRAE\_SAMPLING\_PERIOD & \cellcolor{tabbg1} Enable time-sampling capabilities with the indicated period.\\
  \cellcolor{tabbg2} EXTRAE\_SAMPLING\_VARIABILITY & \cellcolor{tabbg2} Adds some variability to the sampling period.\\
  \cellcolor{tabbg1} EXTRAE\_RUSAGE & \cellcolor{tabbg1} Instrumentation emits resource usage at flush points if set to 1.\\
  \cellcolor{tabbg2} EXTRAE\_SPU\_DMA\_CHANNEL & \cellcolor{tabbg2} Choose the SPU-PPU dma communication channel.\\
  \cellcolor{tabbg1} EXTRAE\_SPU\_BUFFER\_SIZE & \cellcolor{tabbg1} Set the buffer size of the SPU side.\\
  \cellcolor{tabbg2} EXTRAE\_SPU\_FILE\_SIZE & \cellcolor{tabbg2} Set the maximum size for the SPU side (default: 5Mbytes).\\
  \cellcolor{tabbg1} EXTRAE\_TRACE\_TYPE & \cellcolor{tabbg1} Choose whether the resulting tracefiles are intended for Paraver or Dimemas.\\
  \hline
\end{tabular}
}
\caption{Set of environment variables available to configure \TRACE ({\em continued})}
\label{tab:EnvironmentVariables_continued}
\end{table}

\end{landscape}


\chapter{Frequently Asked Questions}\label{sec:FAQ}

%
% Model:
% <bold> Question: <bold> question-text ?

% If there's just one answer
% <bold> Answer  : <bold> answer
% Otherwise
% <bold> Answer1  : <bold> answer1
% <bold> Answer2  : <bold> answer2

\section{Configure, compile and link FAQ}

\begin{itemize}

\item {\bf Question:} The {\tt bootstrap} script claims {\tt libtool} errors like:\\
      {\tt
      src/common/Makefile.am:9: Libtool library used but `LIBTOOL' is undefined\\
      src/common/Makefile.am:9:   The usual way to define `LIBTOOL' is to add `AC\_PROG\_LIBTOOL'\\
      src/common/Makefile.am:9:   to `configure.ac' and run `aclocal' and `autoconf' again.\\
      src/common/Makefile.am:9:   If `AC\_PROG\_LIBTOOL' is in `configure.ac', make sure\\
      src/common/Makefile.am:9:   its definition is in aclocal's search path.\\
      }
      {\bf Answer:  } Add to the {\tt aclocal} (which is called in {\tt bootstrap}) the directory where it can find the M4-macro files from {\tt libtool}. Use the {\tt -I option} to add it.

\item {\bf Question:} The {\tt bootstrap} script claims that some macros are not found in the library, like:\\
      {\tt
      aclocal:configure.ac:338: warning: macro `AM\_PATH\_XML2' not found in library\\
      }
      {\bf Answer:  } Some M4 macros are not found. In this specific example, the libxml2 is not installed or cannot be found in the typical installation directory. To solve this issue, check whether the libxml2 is installed and modify the line in the {\tt bootstrap} script that reads\\
      {\tt 
      \&\& aclocal -I config \ \\
      }
      into\\
      {\tt 
      \&\& aclocal -I config -I/path/to/xml/m4/macros \ \\
      }
      where {\tt /path/to/xml/m4/macros} is the directory where the libxml2 M4 got installed (for example /usr/local/share/aclocal).

\item {\bf Question:} The application cannot be linked succesfully. The link stage complains about (or something similar like)\\
      {\tt ld: 0711-317 ERROR: Undefined symbol: .\_\_udivdi3}.\\
      {\tt ld: 0711-317 ERROR: Undefined symbol: .\_\_mulvsi3}.\\
      {\bf Answer:  } The instrumentation libraries have been compiled with GNU compilers whereas the application is compiled using IBM XL compilers. Add the libgcc\_s library to the link stage of the application. This library can be found under the installation directory of the GNU compiler.

\item {\bf Question:} The make command dies when building libraries belonging \TRACE in an AIX machine with messages like:\\
      {\tt
      libtool: link: ar cru libcommon.a libcommon\_la-utils.o libcommon\_la-events.o\\
      ar: 0707-126 libcommon\_la-utils.o is not valid with the current object file mode.\\
              Use the -X option to specify the desired object mode.\\
      ar: 0707-126 libcommon\_la-events.o is not valid with the current object file mode.\\
              Use the -X option to specify the desired object mode.\\
      }
      {\bf Answer:  } {\tt Libtool} uses {\tt ar} command to build static libraries. However, {\tt ar} does need special flags (-X64) to deal with 64 bit objects. To workaround this problem, just set the environment variable OBJECT\_MODE to 64 before executing {\tt gmake}. The {\tt ar} command honors this variable to properly handle the object files in 64 bit mode.

\item {\bf Question:} The {\tt configure} script dies saying\\
      {\tt configure: error: Unable to determine pthread library support}.\\
      {\bf Answer:  } Some systems (like BG/L) does not provide a pthread library and {\tt configure} claims that cannot find it. Launch the {\tt configure} script with the {\tt -disable-pthread} parameter.

\item {\bf Question:} {NOT!} {\tt gmake} command fails when compiling the instrumentation package in a machine running AIX operating system, using 64 bit mode and IBM XL compilers complaining about Profile MPI (PMPI) symbols.\\
      {\bf Answer:  } {NOT!} Use the reentrant version of IBM compilers ({\tt xlc\_r} and {\tt xlC\_r}). Non reentrant versions of MPI library does not include 64 bit MPI symbols, whereas reentrant versions do. To use these compilers, set the CC (C compiler) and CXX (C++ compiler) environment variables before running the {\tt configure} script.

\item {\bf Question:} The compiler fails complaining that some parameters can not be understand when compiling the parallel merge.
      {\bf Answer:  } If the environment has more than one compiler (for example, IBM and GNU compilers), is it possible that the parallel merge compiler is not the same as the rest of the package. There are two ways to solve this:
      \begin{itemize}
      \item Force the package compilation with the same backend as the parallel compiler. For example, for IBM compiler, set {\tt CC=xlc} and {\tt CXX=xlC} at the configure step.
      \item Tell the parallel compiler to use the same compiler as the rest of the package. For example, for IBM compiler mpcc, set {\tt MP\_COMPILER=gcc} when issuing the make command.
      \end{itemize}

\item {\bf Question:} The instrumentation package does not generate the shared instrumentation libraries but generates the satatic instrumentation libraries.\\
      {\bf Answer 1:} Check that the configure step was compiled without {\tt --disable-shared} or force it to be enabled through {\tt --enable-shared}.\\
      {\bf Answer 2:} Some MPI libraries (like MPICH 1.2.x) do not generate the shared libraries by default. The instrumentation package rely on them to generate its shared libraries, so make sure that the shared libraries of the MPI library are generated.

\end{itemize}

\section{Execution FAQ}

\begin{itemize}

\item {\bf Question:} Why do the environment variables are not exported?\\
      {\bf Answer:  } MPI applications are launched using special programs (like {\tt mpirun, poe, mprun, srun...}) that spawn the application for the selected resources. Some of these programs do not export all the environment variables to the spawned processes. Check if the the launching program does have special parameters to do that, or use the approach used on section \ref{cha:Examples} based on launching scripts instead of MPI applications.

\item {\bf Question:} The application runs but does not generate intermediate trace files (*.mpit)\\
      {\bf Answer 1:} Check that environment variables are correctly passed to the application.\\
      {\bf Answer 2:} If the code is Fortran, check that the number of underscores used to decorate routines in the instrumentation library matches the number of underscores added by the Fortran compiler you used to compile and link the application. You can use the {\tt nm} and {\tt grep} commands to check it. 

\item {\bf Question:} The instrumentation begins for a single process instead for several processes?\\
      {\bf Answer 1:} Check that you place the appropriate parameter to indicate the number of tasks (typically -np).\\
      {\bf Answer 2:} Some MPI implementation require the application to receive special MPI parameters to run correctly. For example, MPICH based on CH-P4 device require the binary to receive som paramters. The following example is an sh-script that solves this issue:\\
      {\tt \#!/bin/sh\\
           EXTRAE\_CONFIG\_FILE=extrae.xml ./mpi\_program \$@ real\_params}\\

\item {\bf Question:} The application blocks at the beginning?\\
      {\bf Answer  :} The application may be waiting for all tasks to startup but only some of them are running. Check for the previous question.

\item {\bf Question:} The resulting traces does not contain the routines that have been instrumented.\\
      {\bf Answer 1:} Check that the routines have been actually executed.\\
      {\bf Answer 2:} Some compilers do automatic inlining of functions at some optimization levels (e.g., Intel Compiler at -O2). When functions are inlined, they do not have entry and exit blocks and cannot be instrumented. Turn off inlining or decrease the optimization level.

\item {\bf Question:} Number of threads = 1?\\
      {\bf Answer  :} Some MPI launchers ({\it i.e.} mpirun, poe, mprun...) do not export all the environment variables to all tasks. Look at chapter \ref{cha:Examples} to workaround this and/or contact your support staff to know how to do it.

\item {\bf Question:} When running the instrumented application, the loader complains about:\\
              {\tt undefined symbol: clock\_gettime}\\
      {\bf Answer  :} The instrumentation package was configured using {\tt --enable-posix-clock} and on many systems this implies the inclusion of additional libraries (namely, {\tt -lrt}).

\end{itemize}

\section{Performance monitoring counters FAQ}

\begin{itemize}

\item {\bf Question:} How do I know the available performance counters on the system?\\
      {\bf Answer 1:} If using PAPI, check the {\tt papi\_avail} or {\tt papi\_native\_avail} commands found in the PAPI installation directory.\\
      {\bf Answer 2:} If using PMAPI (on AIX systems), check for the {\tt pmlist} command. Specifically, check for the available groups running {\tt pmlist -g -1}.

\item {\bf Question:} How do I know how many performance counters can I use?\\
      {\bf Answer:  } The \TRACE package can gather up to eight (8) performance counters at the same time. This also depends on the underlying library used to gather them.

\item {\bf Question:} When using PAPI, I cannot read eight performance counters or the specified in {\tt papi\_avail} output.\\
      {\bf Answer 1:} There are some performance counters (those listed in {\tt papi\_avail}) that are classified as derived. Such performance counters depend on more than one counter increasing the number of real performance counters used. Check for the derived column within the list to check whether a performance counter is derived or not.\\
			{\bf Answer 2:} On some architectures, like the PowerPC, the performance counters are grouped in a such way that choosing a performance counter precludes others from being elected in the same set. A feasible work-around is to create as many sets in the XML file to gather all the required hardware counters and make sure that the sets change from time to time.


\end{itemize}

\section{Merging traces FAQ}

\begin{itemize}

\item {\bf Question:} The {\tt mpi2prv} command shows the following messages at the start-up:\\
      {\tt
      PANIC! Trace file TRACE.0000011148000001000000.mpit is 16 bytes too big!\\
      PANIC! Trace file TRACE.0000011147000002000000.mpit is 32 bytes too big!\\
      PANIC! Trace file TRACE.0000011146000003000000.mpit is 16 bytes too big!\\
      }
      and it dies when parsing the intermediate files.\\
      {\bf Answer  1:} The aforementioned messages are typically related with incomplete writes in disk. Check for enough disk space using the {\tt quota} and {\tt df} commands.
      {\bf Answer  2:} If your system supports multiple ABIs (for example, linux x86-64 supports 32 and 64 bits ABIs), check that the ABI of the target application and the ABI of the merger match.

\item {\bf Question:} The resulting Paraver tracefile contains invalid references to the source code.\\
      {\bf Answer:  } This usually happens when the code has not been compiled and linked with the -g flag. Moreover, some high level optimizations (which includes inlining, interprocedural analysis, and so on) can lead to generate bad references.

\item {\bf Question:} The resulting trace contains information regarding the stack (like callers) but their value does not coincide with the source code.\\
      {\bf Answer:  } Check that the same binary is used to generate the trace and referenced with the the {\tt -e} parameter when generating the Paraver tracefile.

\end{itemize}


\chapter{Instrumented routines}\label{cha:InstrumentedRoutines}

\section{Instrumented MPI routines}\label{sec:MPIinstrumentedroutines}

These are the instrumented MPI routines in the \TRACE package:

\begin{itemize}
\item MPI\_Init
\item MPI\_Init\_thread\footnotemark[1]
\item MPI\_Finalize
\item MPI\_Bsend
\item MPI\_Ssend
\item MPI\_Rsend
\item MPI\_Send
\item MPI\_Bsend\_init
\item MPI\_Ssend\_init
\item MPI\_Rsend\_init
\item MPI\_Send\_init
\item MPI\_Ibsend
\item MPI\_Issend
\item MPI\_Irsend
\item MPI\_Isend
\item MPI\_Recv
\item MPI\_Irecv
\item MPI\_Recv\_init
\item MPI\_Reduce
\item MPI\_Reduce\_scatter
\item MPI\_Allreduce
\item MPI\_Barrier
\item MPI\_Cancel
\item MPI\_Test
\item MPI\_Wait
\item MPI\_Waitall
\item MPI\_Waitany
\item MPI\_Waitsome
\item MPI\_Bcast
\item MPI\_Alltoall
\item MPI\_Alltoallv
\item MPI\_Allgather
\item MPI\_Allgatherv
\item MPI\_Gather
\item MPI\_Gatherv
\item MPI\_Scatter
\item MPI\_Scatterv
\item MPI\_Comm\_rank
\item MPI\_Comm\_size
\item MPI\_Comm\_create
\item MPI\_Comm\_dup
\item MPI\_Comm\_split
\item MPI\_Cart\_create
\item MPI\_Cart\_sub
\item MPI\_Start
\item MPI\_Startall
\item MPI\_Request\_free
\item MPI\_Scan
\item MPI\_Sendrecv
\item MPI\_Sendrecv\_replace
\item MPI\_File\_open\footnotemark[2]
\item MPI\_File\_close\footnotemark[2]
\item MPI\_File\_read\footnotemark[2]
\item MPI\_File\_read\_all\footnotemark[2]
\item MPI\_File\_write\footnotemark[2]
\item MPI\_File\_write\_all\footnotemark[2]
\item MPI\_File\_read\_at\footnotemark[2]
\item MPI\_File\_read\_at\_all\footnotemark[2]
\item MPI\_File\_write\_at\footnotemark[2]
\item MPI\_File\_write\_at\_all\footnotemark[2]
\item MPI\_Get\footnotemark[3]
\item MPI\_Put\footnotemark[3]
\end{itemize}

\footnotetext[1]{The MPI library must support this routine}
\footnotetext[2]{The MPI library must support MPI/IO routines}
\footnotetext[3]{The MPI library must support 1-sided (or RMA -remote memory address-) routines}

\section{Instrumented OpenMP runtimes}\label{sec:OpenMPruntimesinstrumented}

\subsection{Intel compilers - icc, iCC, ifort}

The instrumentation of the Intel OpenMP runtime for versions 8.1 to 10.1 is only available using the \TRACE package based on DynInst library.

These are the instrument routines of the Intel OpenMP runtime functions using DynInst:

\begin{itemize}
\item \_\_kmpc\_fork\_call
\item \_\_kmpc\_barrier
\item \_\_kmpc\_invoke\_task\_func
\item \_\_kmpc\_set\_lock\footnotemark[4]
\item \_\_kmpc\_unset\_lock\footnotemark[4]
\end{itemize}
\footnotetext[4]{The instrumentation of OpenMP locks can be enabled/disabled}

The instrumentation of the Intel OpenMP runtime for version 11.0 to 12.0 is available using the \TRACE package based on the {\tt LD\_PRELOAD} and also the DynInst mechanisms. The instrumented routines include:

\begin{itemize}
\item \_\_kmpc\_fork\_call
\item \_\_kmpc\_barrier
\item \_\_kmpc\_dispatch\_next\_4
\item \_\_kmpc\_dispatch\_next\_8
\item \_\_kmpc\_single
\item \_\_kmpc\_end\_single
\item \_\_kmpc\_critical\footnotemark[4]
\item \_\_kmpc\_end\_critical\footnotemark[4]
\item omp\_set\_lock\footnotemark[4]
\item omp\_unset\_lock\footnotemark[4]
\item \_\_kmpc\_omp\_task\_alloc
\item \_\_kmpc\_omp\_task\_begin\_if0
\item \_\_kmpc\_omp\_task\_complete\_if0
\item \_\_kmpc\_omp\_taskwait
\end{itemize}

\subsection{IBM compilers - xlc, xlC, xlf}

\TRACE supports IBM OpenMP runtime 1.6.

These are the instrumented routines of the IBM OpenMP runtime:

\begin{itemize}
\item \_xlsmpParallelDoSetup\_TPO
\item \_xlsmpParRegionSetup\_TPO
\item \_xlsmpWSDoSetup\_TPO
\item \_xlsmpBarrier\_TPO
\item \_xlsmpSingleSetup\_TPO
\item \_xlsmpWSSectSetup\_TPO
\item \_xlsmpRelDefaultSLock\footnotemark[4]
\item \_xlsmpGetDefaultSLock\footnotemark[4]
\end{itemize}

\subsection{GNU compilers - gcc, g++, gfortran}

\TRACE supports GNU OpenMP runtime 4.2.

These are the instrumented routines of the IBM OpenMP runtime:

\begin{itemize}
\item GOMP\_parallel\_start
\item GOMP\_parallel\_sections\_start
\item GOMP\_parallel\_end
\item GOMP\_sections\_start
\item GOMP\_sections\_next
\item GOMP\_sections\_end
\item GOMP\_sections\_end\_nowait
\item GOMP\_loop\_end
\item GOMP\_loop\_end\_nowait
\item GOMP\_loop\_static\_start
\item GOMP\_loop\_dynamic\_start
\item GOMP\_loop\_guided\_start
\item GOMP\_loop\_runtime\_start
\item GOMP\_parallel\_loop\_static\_start
\item GOMP\_parallel\_loop\_dynamic\_start
\item GOMP\_parallel\_loop\_guided\_start
\item GOMP\_parallel\_loop\_runtime\_start
\item GOMP\_loop\_static\_next
\item GOMP\_loop\_dynamic\_next
\item GOMP\_loop\_guided\_next
\item GOMP\_loop\_runtime\_next
\item GOMP\_barrier
\item GOMP\_critical\_name\_enter\footnotemark[4]
\item GOMP\_critical\_name\_exit\footnotemark[4]
\item GOMP\_critical\_enter\footnotemark[4]
\item GOMP\_critical\_exit\footnotemark[4]
\item GOMP\_atomic\_enter\footnotemark[4]
\item GOMP\_atomic\_exit\footnotemark[4]
\item GOMP\_task
\item GOMP\_taskwait
\end{itemize}

\section{Instrumented pthread runtimes}\label{sec:OpenMPruntimesinstrumented}

These are the instrumented routines of the pthread runtime:

\begin{itemize}
\item pthread\_create
\item pthread\_detach
\item pthread\_join
\item pthread\_barrier\_wait
\item pthread\_mutex\_lock
\item pthread\_mutex\_trylock
\item pthread\_mutex\_timedlock
\item pthread\_mutex\_unlock
% pthread_cond_* routines seem to be not instrumentable. the application hangs when instrumenting them
%\item pthread\_cond\_signal
%\item pthread\_cond\_broadcast
%\item pthread\_cond\_wait
%\item pthread\_cond\_timedwait
\item pthread\_rwlock\_rdlock
\item pthread\_rwlock\_tryrdlock
\item pthread\_rwlock\_timedrdlock
\item pthread\_rwlock\_wrlock
\item pthread\_rwlock\_trywrlock
\item pthread\_rwlock\_timedwrlock
\item pthread\_rwlock\_unlock
\end{itemize}


\end{document}

