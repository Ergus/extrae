\chapter{Frequently Asked Questions}\label{sec:FAQ}

%
% Model:
% <bold> Question: <bold> question-text ?

% If there's just one answer
% <bold> Answer  : <bold> answer
% Otherwise
% <bold> Answer1  : <bold> answer1
% <bold> Answer2  : <bold> answer2

\section{Configure, compile and link FAQ}

\begin{itemize}

\item {\bf Question:} The {\tt bootstrap} script claims {\tt libtool} errors like:\\
      {\tt
      src/common/Makefile.am:9: Libtool library used but `LIBTOOL' is undefined\\
      src/common/Makefile.am:9:   The usual way to define `LIBTOOL' is to add `AC\_PROG\_LIBTOOL'\\
      src/common/Makefile.am:9:   to `configure.ac' and run `aclocal' and `autoconf' again.\\
      src/common/Makefile.am:9:   If `AC\_PROG\_LIBTOOL' is in `configure.ac', make sure\\
      src/common/Makefile.am:9:   its definition is in aclocal's search path.\\
      }
      {\bf Answer:  } Add to the {\tt aclocal} (which is called in {\tt bootstrap}) the directory where it can find the M4-macro files from {\tt libtool}. Use the {\tt -I option} to add it.

\item {\bf Question:} The {\tt bootstrap} script claims that some macros are not found in the library, like:\\
      {\tt
      aclocal:configure.ac:338: warning: macro `AM\_PATH\_XML2' not found in library\\
      }
      {\bf Answer:  } Some M4 macros are not found. In this specific example, the libxml2 is not installed or cannot be found in the typical installation directory. To solve this issue, check whether the libxml2 is installed and modify the line in the {\tt bootstrap} script that reads\\
      {\tt 
      \&\& aclocal -I config \ \\
      }
      into\\
      {\tt 
      \&\& aclocal -I config -I/path/to/xml/m4/macros \ \\
      }
      where {\tt /path/to/xml/m4/macros} is the directory where the libxml2 M4 got installed (for example /usr/local/share/aclocal).

\item {\bf Question:} The application cannot be linked succesfully. The link stage complains about (or something similar like)\\
      {\tt ld: 0711-317 ERROR: Undefined symbol: .\_\_udivdi3}.\\
      {\tt ld: 0711-317 ERROR: Undefined symbol: .\_\_mulvsi3}.\\
      {\bf Answer:  } The instrumentation libraries have been compiled with GNU compilers whereas the application is compiled using IBM XL compilers. Add the libgcc\_s library to the link stage of the application. This library can be found under the installation directory of the GNU compiler.

\item {\bf Question:} The application cannot be linked. The linker misses some routines like\\
      {\tt src/common/utils.c:122: undefined reference to `\_\_intel\_sse2\_strlen'}\\
      {\tt src/common/utils.c:125: undefined reference to `\_\_intel\_sse2\_strdup'}\\
      {\tt src/common/utils.c:132: undefined reference to `\_\_intel\_sse2\_strtok'}\\
      {\tt src/common/utils.c:100: undefined reference to `\_\_intel\_sse2\_strncpy'}\\
      {\tt src/common/timesync.c:211: undefined reference to `\_intel\_fast\_memset'}\\
      {\bf Answer:  } The instrumentation libraries have been compiled using Intel compilers (i.e. {\tt icc}, {\tt icpc}) whereas the application is being linked through non-Intel compilers or {\tt ld} directly. You can proceed in three directions, you can either compile your application using the Intel compilers, or add a Intel library that provides these routines ({\tt libintlc.so} and {\tt libirc.so}, for instance), or even recompile \TRACE using the GNU compilers. Note, moreover, that using Intel MPI compiler does not guarantee using the Intel compiler backends, just run the MPI compiler ({\tt mpicc}, {\tt mpiCC}, {\tt mpif77}, {\tt mpif90}, .. ) with the {\tt -v} flag to get information on what compiler backend relies.

\item {\bf Question:} The make command dies when building libraries belonging \TRACE in an AIX machine with messages like:\\
      {\tt
      libtool: link: ar cru libcommon.a libcommon\_la-utils.o libcommon\_la-events.o\\
      ar: 0707-126 libcommon\_la-utils.o is not valid with the current object file mode.\\
              Use the -X option to specify the desired object mode.\\
      ar: 0707-126 libcommon\_la-events.o is not valid with the current object file mode.\\
              Use the -X option to specify the desired object mode.\\
      }
      {\bf Answer:  } {\tt Libtool} uses {\tt ar} command to build static libraries. However, {\tt ar} does need special flags (-X64) to deal with 64 bit objects. To workaround this problem, just set the environment variable OBJECT\_MODE to 64 before executing {\tt gmake}. The {\tt ar} command honors this variable to properly handle the object files in 64 bit mode.

\item {\bf Question:} The {\tt configure} script dies saying\\
      {\tt configure: error: Unable to determine pthread library support}.\\
      {\bf Answer:  } Some systems (like BG/L) does not provide a pthread library and {\tt configure} claims that cannot find it. Launch the {\tt configure} script with the {\tt -disable-pthread} parameter.

\item {\bf Question:} {NOT!} {\tt gmake} command fails when compiling the instrumentation package in a machine running AIX operating system, using 64 bit mode and IBM XL compilers complaining about Profile MPI (PMPI) symbols.\\
      {\bf Answer:  } {NOT!} Use the reentrant version of IBM compilers ({\tt xlc\_r} and {\tt xlC\_r}). Non reentrant versions of MPI library does not include 64 bit MPI symbols, whereas reentrant versions do. To use these compilers, set the CC (C compiler) and CXX (C++ compiler) environment variables before running the {\tt configure} script.

\item {\bf Question:} The compiler fails complaining that some parameters can not be understand when compiling the parallel merge.
      {\bf Answer:  } If the environment has more than one compiler (for example, IBM and GNU compilers), is it possible that the parallel merge compiler is not the same as the rest of the package. There are two ways to solve this:
      \begin{itemize}
      \item Force the package compilation with the same backend as the parallel compiler. For example, for IBM compiler, set {\tt CC=xlc} and {\tt CXX=xlC} at the configure step.
      \item Tell the parallel compiler to use the same compiler as the rest of the package. For example, for IBM compiler mpcc, set {\tt MP\_COMPILER=gcc} when issuing the make command.
      \end{itemize}

\item {\bf Question:} The instrumentation package does not generate the shared instrumentation libraries but generates the satatic instrumentation libraries.\\
      {\bf Answer 1:} Check that the configure step was compiled without {\tt --disable-shared} or force it to be enabled through {\tt --enable-shared}.\\
      {\bf Answer 2:} Some MPI libraries (like MPICH 1.2.x) do not generate the shared libraries by default. The instrumentation package rely on them to generate its shared libraries, so make sure that the shared libraries of the MPI library are generated.

\item {\bf Question:} In BlueGene systems where the libxml2 (or any optional library for extrae) the linker shows error messages like when compiling the final application with the \TRACE library:\\
			{\tt
			../libxml2/lib/libxml2.a(xmlschemastypes.o): In function `\_xmlSchemaDateAdd':\\
			../libxml2-2.7.2/xmlschemastypes.c:3771: undefined reference to `\_\_uitrunc'\\
			../libxml2-2.7.2/xmlschemastypes.c:3796: undefined reference to `\_\_uitrunc'\\
			../libxml2-2.7.2/xmlschemastypes.c:3801: undefined reference to `\_\_uitrunc'\\
			../libxml2-2.7.2/xmlschemastypes.c:3842: undefined reference to `\_\_uitrunc'\\
			../libxml2-2.7.2/xmlschemastypes.c:3843: undefined reference to `\_\_uitrunc'\\
			../libxml2/lib/libxml2.a(xmlschemastypes.o): In function `xmlSchemaGetCanonValue':\\
			../libxml2-2.7.2/xmlschemastypes.c:5840: undefined reference to `\_\_f64tou64rz'\\
			../libxml2-2.7.2/xmlschemastypes.c:5843: undefined reference to `\_\_f64tou64rz'\\
			../libxml2-2.7.2/xmlschemastypes.c:5846: undefined reference to `\_\_f64tou64rz'\\
			../libxml2-2.7.2/xmlschemastypes.c:5849: undefined reference to `\_\_f64tou64rz'\\
			../libxml2/lib/libxml2.a(debugXML.o): In function `xmlShell':\\
			../libxml2-2.7.2/debugXML.c:2802: undefined reference to `\_fill'\\
			collect2: ld returned 1 exit status\\
			}
			{\bf Answer:  } The libxml2 library (or any other optional library) has been compiled using the IBM XL compiler. There are two alternatives to circumvent the problem: add the XL libraries into the link stage when building your application, or recompile the libxml2 library using the GNU gcc cross compiler for BlueGene.

\item {\bf Question:} Where do I get the procedure and constant declarations for Fortran?\\
      {\bf Answer:  } You can find a module (ready to be compiled) in {\tt \${EXTRAE\_HOME}/include/extrae\_module.f}. To use the module, just compile it (do not link it), and then use it in your compiling / linking step. If you do not use the module, the trace generation (specially for those routines that expect parameters which are not {\tt INTEGER*4}) can result in type errors and thus generate a tracefile that does not honor the \TRACE calls.

\end{itemize}

\section{Execution FAQ}

\begin{itemize}

\item {\bf Question:} I executed my application instrumenting with Extrae, even though it appears that Extrae is not intrumenting anything. There is neither any Extrae message nor any Extrae output files (set-X/*.mpit)\\
      {\bf Answer 1:} Check that environment variables are correctly passed to the application.\\
      {\bf Answer 2:} If the code is Fortran, check that the number of underscores used to decorate routines in the instrumentation library matches the number of underscores added by the Fortran compiler you used to compile and link the application. You can use the {\tt nm} and {\tt grep} commands to check it.\\
      {\bf Answer 3:} If the code is MPI and Fortran, check that you're using the proper Fortran library for the instrumentation.\\
      {\bf Answer 4:} If the code is MPI and you are using LD\_PRELOAD, check that the binary is linked against a shared MPI library (you can use the {\tt ldd} command).\\

\item {\bf Question:} Why do the environment variables are not exported?\\
      {\bf Answer:  } MPI applications are launched using special programs (like {\tt mpirun, poe, mprun, srun...}) that spawn the application for the selected resources. Some of these programs do not export all the environment variables to the spawned processes. Check if the the launching program does have special parameters to do that, or use the approach used on section \ref{cha:Examples} based on launching scripts instead of MPI applications.

\item {\bf Question:} The instrumentation begins for a single process instead for several processes?\\
      {\bf Answer 1:} Check that you place the appropriate parameter to indicate the number of tasks (typically -np).\\
      {\bf Answer 2:} Some MPI implementation require the application to receive special MPI parameters to run correctly. For example, MPICH based on CH-P4 device require the binary to receive som paramters. The following example is an sh-script that solves this issue:\\
      {\tt \#!/bin/sh\\
           EXTRAE\_CONFIG\_FILE=extrae.xml ./mpi\_program \$@ real\_params}\\

\item {\bf Question:} The application blocks at the beginning?\\
      {\bf Answer  :} The application may be waiting for all tasks to startup but only some of them are running. Check for the previous question.

\item {\bf Question:} The resulting traces does not contain the routines that have been instrumented.\\
      {\bf Answer 1:} Check that the routines have been actually executed.\\
      {\bf Answer 2:} Some compilers do automatic inlining of functions at some optimization levels (e.g., Intel Compiler at -O2). When functions are inlined, they do not have entry and exit blocks and cannot be instrumented. Turn off inlining or decrease the optimization level.

\item {\bf Question:} Number of threads = 1?\\
      {\bf Answer  :} Some MPI launchers ({\it i.e.} mpirun, poe, mprun...) do not export all the environment variables to all tasks. Look at chapter \ref{cha:Examples} to workaround this and/or contact your support staff to know how to do it.

\item {\bf Question:} When running the instrumented application, the loader complains about:\\
              {\tt undefined symbol: clock\_gettime}\\
      {\bf Answer  :} The instrumentation package was configured using {\tt --enable-posix-clock} and on many systems this implies the inclusion of additional libraries (namely, {\tt -lrt}).

\end{itemize}

\section{Performance monitoring counters FAQ}

\begin{itemize}

\item {\bf Question:} How do I know the available performance counters on the system?\\
      {\bf Answer 1:} If using PAPI, check the {\tt papi\_avail} or {\tt papi\_native\_avail} commands found in the PAPI installation directory.\\
      {\bf Answer 2:} If using PMAPI (on AIX systems), check for the {\tt pmlist} command. Specifically, check for the available groups running {\tt pmlist -g -1}.

\item {\bf Question:} How do I know how many performance counters can I use?\\
      {\bf Answer:  } The \TRACE package can gather up to eight (8) performance counters at the same time. This also depends on the underlying library used to gather them.

\item {\bf Question:} When using PAPI, I cannot read eight performance counters or the specified in {\tt papi\_avail} output.\\
      {\bf Answer 1:} There are some performance counters (those listed in {\tt papi\_avail}) that are classified as derived. Such performance counters depend on more than one counter increasing the number of real performance counters used. Check for the derived column within the list to check whether a performance counter is derived or not.\\
			{\bf Answer 2:} On some architectures, like the PowerPC, the performance counters are grouped in a such way that choosing a performance counter precludes others from being elected in the same set. A feasible work-around is to create as many sets in the XML file to gather all the required hardware counters and make sure that the sets change from time to time.


\end{itemize}

\section{Merging traces FAQ}

\begin{itemize}

\item {\bf Question:} The {\tt mpi2prv} command shows the following messages at the start-up:\\
      {\tt
      PANIC! Trace file TRACE.0000011148000001000000.mpit is 16 bytes too big!\\
      PANIC! Trace file TRACE.0000011147000002000000.mpit is 32 bytes too big!\\
      PANIC! Trace file TRACE.0000011146000003000000.mpit is 16 bytes too big!\\
      }
      and it dies when parsing the intermediate files.\\
      {\bf Answer  1:} The aforementioned messages are typically related with incomplete writes in disk. Check for enough disk space using the {\tt quota} and {\tt df} commands.
      {\bf Answer  2:} If your system supports multiple ABIs (for example, linux x86-64 supports 32 and 64 bits ABIs), check that the ABI of the target application and the ABI of the merger match.

\item {\bf Question:} The resulting Paraver tracefile contains invalid references to the source code.\\
      {\bf Answer:  } This usually happens when the code has not been compiled and linked with the -g flag. Moreover, some high level optimizations (which includes inlining, interprocedural analysis, and so on) can lead to generate bad references.

\item {\bf Question:} The resulting trace contains information regarding the stack (like callers) but their value does not coincide with the source code.\\
      {\bf Answer:  } Check that the same binary is used to generate the trace and referenced with the the {\tt -e} parameter when generating the Paraver tracefile.

\end{itemize}
