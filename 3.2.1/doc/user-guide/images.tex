\batchmode
\documentclass[twoside,a4,english,11pt]{book}
\RequirePackage{ifthen}




\usepackage{hyperref}
\usepackage{fontenc}
\usepackage[latin9]{inputenc}
\usepackage{url}
\usepackage{epsfig} 
\usepackage{subfigure} 
\usepackage{setspace} 
\usepackage{tocbibind} 
\usepackage{a4wide}
\usepackage{fullpage}
\usepackage{verbatim}
\usepackage{color,calc}
\usepackage{pdflscape}
\usepackage{fancyvrb}
\usepackage{amsfonts}
\usepackage{multirow}
\usepackage[table]{xcolor}


\definecolor{lightgray}{rgb}{0.75,0.75,0.75}
\definecolor{tabbg1}{rgb}{0.85,0.85,0.85}
\definecolor{tabfg1}{rgb}{0.0,0.0,0.0}
\definecolor{tabbg2}{rgb}{0.95,0.95,0.95}
\definecolor{tabfg2}{rgb}{0.0,0.0,0.0}


\urlstyle{leostyle}


\makeatletter

\usepackage{babel}
\makeatother


\pagestyle{plain}

%
\providecommand{\TRACE}{{\sf {E}xtrae}\ }%
\providecommand{\PARAVER}{{\sf Paraver}\ }%
\providecommand{\TRACEVERSION}{3.1.1rc} 





\pagecolor[gray]{.7}

\usepackage[latin9]{inputenc}



\makeatletter
\AtBeginDocument{\makeatletter
\input /home/harald/development/extrae/trunk/doc/user-guide.aux
\makeatother
}

\makeatletter
\count@=\the\catcode`\_ \catcode`\_=8 
\newenvironment{tex2html_wrap}{}{}%
\catcode`\<=12\catcode`\_=\count@
\newcommand{\providedcommand}[1]{\expandafter\providecommand\csname #1\endcsname}%
\newcommand{\renewedcommand}[1]{\expandafter\providecommand\csname #1\endcsname{}%
  \expandafter\renewcommand\csname #1\endcsname}%
\newcommand{\newedenvironment}[1]{\newenvironment{#1}{}{}\renewenvironment{#1}}%
\let\newedcommand\renewedcommand
\let\renewedenvironment\newedenvironment
\makeatother
\let\mathon=$
\let\mathoff=$
\ifx\AtBeginDocument\undefined \newcommand{\AtBeginDocument}[1]{}\fi
\newbox\sizebox
\setlength{\hoffset}{0pt}\setlength{\voffset}{0pt}
\addtolength{\textheight}{\footskip}\setlength{\footskip}{0pt}
\addtolength{\textheight}{\topmargin}\setlength{\topmargin}{0pt}
\addtolength{\textheight}{\headheight}\setlength{\headheight}{0pt}
\addtolength{\textheight}{\headsep}\setlength{\headsep}{0pt}
\setlength{\textwidth}{349pt}
\newwrite\lthtmlwrite
\makeatletter
\let\realnormalsize=\normalsize
\global\topskip=2sp
\def\preveqno{}\let\real@float=\@float \let\realend@float=\end@float
\def\@float{\let\@savefreelist\@freelist\real@float}
\def\liih@math{\ifmmode$\else\bad@math\fi}
\def\end@float{\realend@float\global\let\@freelist\@savefreelist}
\let\real@dbflt=\@dbflt \let\end@dblfloat=\end@float
\let\@largefloatcheck=\relax
\let\if@boxedmulticols=\iftrue
\def\@dbflt{\let\@savefreelist\@freelist\real@dbflt}
\def\adjustnormalsize{\def\normalsize{\mathsurround=0pt \realnormalsize
 \parindent=0pt\abovedisplayskip=0pt\belowdisplayskip=0pt}%
 \def\phantompar{\csname par\endcsname}\normalsize}%
\def\lthtmltypeout#1{{\let\protect\string \immediate\write\lthtmlwrite{#1}}}%
\newcommand\lthtmlhboxmathA{\adjustnormalsize\setbox\sizebox=\hbox\bgroup\kern.05em }%
\newcommand\lthtmlhboxmathB{\adjustnormalsize\setbox\sizebox=\hbox to\hsize\bgroup\hfill }%
\newcommand\lthtmlvboxmathA{\adjustnormalsize\setbox\sizebox=\vbox\bgroup %
 \let\ifinner=\iffalse \let\)\liih@math }%
\newcommand\lthtmlboxmathZ{\@next\next\@currlist{}{\def\next{\voidb@x}}%
 \expandafter\box\next\egroup}%
\newcommand\lthtmlmathtype[1]{\gdef\lthtmlmathenv{#1}}%
\newcommand\lthtmllogmath{\dimen0\ht\sizebox \advance\dimen0\dp\sizebox
  \ifdim\dimen0>.95\vsize
   \lthtmltypeout{%
*** image for \lthtmlmathenv\space is too tall at \the\dimen0, reducing to .95 vsize ***}%
   \ht\sizebox.95\vsize \dp\sizebox\z@ \fi
  \lthtmltypeout{l2hSize %
:\lthtmlmathenv:\the\ht\sizebox::\the\dp\sizebox::\the\wd\sizebox.\preveqno}}%
\newcommand\lthtmlfigureA[1]{\let\@savefreelist\@freelist
       \lthtmlmathtype{#1}\lthtmlvboxmathA}%
\newcommand\lthtmlpictureA{\bgroup\catcode`\_=8 \lthtmlpictureB}%
\newcommand\lthtmlpictureB[1]{\lthtmlmathtype{#1}\egroup
       \let\@savefreelist\@freelist \lthtmlhboxmathB}%
\newcommand\lthtmlpictureZ[1]{\hfill\lthtmlfigureZ}%
\newcommand\lthtmlfigureZ{\lthtmlboxmathZ\lthtmllogmath\copy\sizebox
       \global\let\@freelist\@savefreelist}%
\newcommand\lthtmldisplayA{\bgroup\catcode`\_=8 \lthtmldisplayAi}%
\newcommand\lthtmldisplayAi[1]{\lthtmlmathtype{#1}\egroup\lthtmlvboxmathA}%
\newcommand\lthtmldisplayB[1]{\edef\preveqno{(\theequation)}%
  \lthtmldisplayA{#1}\let\@eqnnum\relax}%
\newcommand\lthtmldisplayZ{\lthtmlboxmathZ\lthtmllogmath\lthtmlsetmath}%
\newcommand\lthtmlinlinemathA{\bgroup\catcode`\_=8 \lthtmlinlinemathB}
\newcommand\lthtmlinlinemathB[1]{\lthtmlmathtype{#1}\egroup\lthtmlhboxmathA
  \vrule height1.5ex width0pt }%
\newcommand\lthtmlinlineA{\bgroup\catcode`\_=8 \lthtmlinlineB}%
\newcommand\lthtmlinlineB[1]{\lthtmlmathtype{#1}\egroup\lthtmlhboxmathA}%
\newcommand\lthtmlinlineZ{\egroup\expandafter\ifdim\dp\sizebox>0pt %
  \expandafter\centerinlinemath\fi\lthtmllogmath\lthtmlsetinline}
\newcommand\lthtmlinlinemathZ{\egroup\expandafter\ifdim\dp\sizebox>0pt %
  \expandafter\centerinlinemath\fi\lthtmllogmath\lthtmlsetmath}
\newcommand\lthtmlindisplaymathZ{\egroup %
  \centerinlinemath\lthtmllogmath\lthtmlsetmath}
\def\lthtmlsetinline{\hbox{\vrule width.1em \vtop{\vbox{%
  \kern.1em\copy\sizebox}\ifdim\dp\sizebox>0pt\kern.1em\else\kern.3pt\fi
  \ifdim\hsize>\wd\sizebox \hrule depth1pt\fi}}}
\def\lthtmlsetmath{\hbox{\vrule width.1em\kern-.05em\vtop{\vbox{%
  \kern.1em\kern0.8 pt\hbox{\hglue.17em\copy\sizebox\hglue0.8 pt}}\kern.3pt%
  \ifdim\dp\sizebox>0pt\kern.1em\fi \kern0.8 pt%
  \ifdim\hsize>\wd\sizebox \hrule depth1pt\fi}}}
\def\centerinlinemath{%
  \dimen1=\ifdim\ht\sizebox<\dp\sizebox \dp\sizebox\else\ht\sizebox\fi
  \advance\dimen1by.5pt \vrule width0pt height\dimen1 depth\dimen1 
 \dp\sizebox=\dimen1\ht\sizebox=\dimen1\relax}

\def\lthtmlcheckvsize{\ifdim\ht\sizebox<\vsize 
  \ifdim\wd\sizebox<\hsize\expandafter\hfill\fi \expandafter\vfill
  \else\expandafter\vss\fi}%
\providecommand{\selectlanguage}[1]{}%
\makeatletter \tracingstats = 1 


\begin{document}
\pagestyle{empty}\thispagestyle{empty}\lthtmltypeout{}%
\lthtmltypeout{latex2htmlLength hsize=\the\hsize}\lthtmltypeout{}%
\lthtmltypeout{latex2htmlLength vsize=\the\vsize}\lthtmltypeout{}%
\lthtmltypeout{latex2htmlLength hoffset=\the\hoffset}\lthtmltypeout{}%
\lthtmltypeout{latex2htmlLength voffset=\the\voffset}\lthtmltypeout{}%
\lthtmltypeout{latex2htmlLength topmargin=\the\topmargin}\lthtmltypeout{}%
\lthtmltypeout{latex2htmlLength topskip=\the\topskip}\lthtmltypeout{}%
\lthtmltypeout{latex2htmlLength headheight=\the\headheight}\lthtmltypeout{}%
\lthtmltypeout{latex2htmlLength headsep=\the\headsep}\lthtmltypeout{}%
\lthtmltypeout{latex2htmlLength parskip=\the\parskip}\lthtmltypeout{}%
\lthtmltypeout{latex2htmlLength oddsidemargin=\the\oddsidemargin}\lthtmltypeout{}%
\makeatletter
\if@twoside\lthtmltypeout{latex2htmlLength evensidemargin=\the\evensidemargin}%
\else\lthtmltypeout{latex2htmlLength evensidemargin=\the\oddsidemargin}\fi%
\lthtmltypeout{}%
\makeatother
\setcounter{page}{1}
\onecolumn

% !!! IMAGES START HERE !!!

\stepcounter{chapter}
\stepcounter{section}
\stepcounter{subsection}
\stepcounter{subsection}
\stepcounter{section}
\stepcounter{subsection}
\stepcounter{subsection}
\stepcounter{section}
\stepcounter{chapter}
\stepcounter{chapter}
\stepcounter{section}
\stepcounter{section}
\stepcounter{section}
\stepcounter{section}
\stepcounter{section}
\stepcounter{subsection}
\stepcounter{subsection}
\stepcounter{subsection}
\stepcounter{subsubsection}
\stepcounter{subsubsection}
\stepcounter{subsection}
\stepcounter{subsubsection}
\stepcounter{subsubsection}
\stepcounter{subsubsection}
\stepcounter{subsubsection}
\stepcounter{subsubsection}
\stepcounter{subsubsection}
\stepcounter{subsubsection}
\stepcounter{subsubsection}
\stepcounter{subsubsection}
\stepcounter{subsubsection}
\stepcounter{subsubsection}
\stepcounter{subsubsection}
\stepcounter{subsubsection}
\stepcounter{subsubsection}
\stepcounter{section}
\stepcounter{chapter}
\stepcounter{section}
\stepcounter{section}
\stepcounter{section}
\stepcounter{section}
\stepcounter{section}
\stepcounter{section}
\stepcounter{section}
\stepcounter{section}
\stepcounter{section}
\stepcounter{subsection}
\stepcounter{subsection}
\stepcounter{subsection}
\stepcounter{section}
\stepcounter{section}
\stepcounter{section}
\stepcounter{section}
\stepcounter{section}
\stepcounter{section}
\stepcounter{section}
\stepcounter{section}
\stepcounter{section}
\stepcounter{section}
\stepcounter{section}
\stepcounter{section}
\stepcounter{section}
\stepcounter{chapter}
\stepcounter{section}
\stepcounter{section}
\stepcounter{section}
\stepcounter{subsection}
\stepcounter{subsection}
\stepcounter{section}
\stepcounter{subsection}
\stepcounter{chapter}
\stepcounter{section}
\stepcounter{subsection}
\stepcounter{subsection}
\stepcounter{section}
\stepcounter{section}
\stepcounter{subsection}
\stepcounter{subsubsection}
\stepcounter{subsubsection}
\stepcounter{subsection}
\stepcounter{subsubsection}
\stepcounter{chapter}
\stepcounter{section}
\stepcounter{section}
\stepcounter{subsection}
\stepcounter{subsection}
\stepcounter{subsection}
\stepcounter{section}
\stepcounter{subsection}
\stepcounter{subsection}
\stepcounter{subsection}
\stepcounter{chapter}
\stepcounter{section}
\stepcounter{subsection}
\stepcounter{subsection}
\stepcounter{section}
\stepcounter{subsection}
\stepcounter{subsection}
\stepcounter{subsection}
\stepcounter{section}
\stepcounter{subsection}
\stepcounter{subsection}
\stepcounter{section}
\appendix
\stepcounter{chapter}
\stepcounter{chapter}
{\newpage\clearpage
\lthtmlfigureA{landscape4474}%
\begin{landscape}
% latex2html id marker 4474

\begin{table}
\centerline{
\small
\begin{tabular}{p{7.5cm} p{14cm}}
  \hline
  {\bf Environment variable} & {\bf Description}\\
  \hline
  \cellcolor{tabbg2} & \cellcolor{tabbg2} Set the number of records that the instrumentation buffer can\\
  \cellcolor{tabbg2}\multirow{-2}{*}{EXTRAE\_BUFFER\_SIZE} & \cellcolor{tabbg2} hold before flushing them.\\
  \cellcolor{tabbg1} & \cellcolor{tabbg1}See section \ref{subsec:ProcessorPerformanceCounters}. Just one set can be defined. Counters (in PAPI)\\
  \cellcolor{tabbg1}\multirow{-2}{*}{EXTRAE\_COUNTERS} & \cellcolor{tabbg1}groups (in PMAPI) are given separated by commas.\\
  \cellcolor{tabbg2}EXTRAE\_CONTROL\_FILE & \cellcolor{tabbg2}The instrumentation will be enabled only when the file pointed exists.\\
  \cellcolor{tabbg1} & \cellcolor{tabbg1}Starts the instrumentation when the specified number of global collectives\\
  \cellcolor{tabbg1}\multirow{-2}{*}{EXTRAE\_CONTROL\_GLOPS} & \cellcolor{tabbg1}have been executed.\\
  \cellcolor{tabbg2}EXTRAE\_CONTROL\_TIME & \cellcolor{tabbg2}Checks the file pointed by {\tt EXTRAE\_CONTROL\_FILE} at this period.\\
  \cellcolor{tabbg1} & \cellcolor{tabbg1}Specifies where temporal files will be created during\\
  \cellcolor{tabbg1}\multirow{-2}{*}{EXTRAE\_DIR} & \cellcolor{tabbg1}instrumentation.\\
  \cellcolor{tabbg2}EXTRAE\_DISABLE\_MPI & \cellcolor{tabbg2}Disable MPI instrumentation.\\
  \cellcolor{tabbg1}EXTRAE\_DISABLE\_OMP & \cellcolor{tabbg1}Disable OpenMP instrumentation.\\
  \cellcolor{tabbg2}EXTRAE\_DISABLE\_PTHREAD & \cellcolor{tabbg2}Disable pthread instrumentation.\\
  \cellcolor{tabbg1}EXTRAE\_DISABLE\_PACX & \cellcolor{tabbg1}Disable PACX instrumentation.\\
  \cellcolor{tabbg2}EXTRAE\_FILE\_SIZE & \cellcolor{tabbg2}Set the maximum size (in Mbytes) for the intermediate trace file.\\
  \cellcolor{tabbg1}                  & \cellcolor{tabbg1}List of routine to be instrumented, as described in \ref{sec:XMLSectionUF} using the\\
  \cellcolor{tabbg1}EXTRAE\_FUNCTIONS & \cellcolor{tabbg1}GNU C {\tt -finstrument-functions} or the IBM XL {\tt -qdebug=function\_trace}\\
  \cellcolor{tabbg1}                  & \cellcolor{tabbg1}option at compile and link time. \\
  \cellcolor{tabbg2} & \cellcolor{tabbg2}Specify if the performance counters should be collected when a\\
  \cellcolor{tabbg2}\multirow{-2}{*}{EXTRAE\_FUNCTIONS\_COUNTERS\_ON} & \cellcolor{tabbg2}user function event is emitted.\\
  \cellcolor{tabbg1}EXTRAE\_FINAL\_DIR & \cellcolor{tabbg1}Specifies where files will be stored when the application ends.\\
  \cellcolor{tabbg2} & \cellcolor{tabbg2}Gather intermediate trace files into a single directory\\
  \cellcolor{tabbg2}\multirow{-2}{*}{EXTRAE\_GATHER\_MPITS} & (\em{this is only available when instrumenting MPI applications}).\\
  \cellcolor{tabbg1}EXTRAE\_HOME & \cellcolor{tabbg1}Points where the {\sf {E}xtrae}\ is installed.\\
  \cellcolor{tabbg2} & \cellcolor{tabbg2}Choose whether the instrumentation runs in in {\tt detail} or\\
  \cellcolor{tabbg2}\multirow{-2}{*}{EXTRAE\_INITIAL\_MODE} & \cellcolor{tabbg2}in {\tt bursts} mode.\\
  \cellcolor{tabbg1}EXTRAE\_BURST\_THRESHOLD & \cellcolor{tabbg1}Specify the threshold time to filter running bursts.\\
  \cellcolor{tabbg2}EXTRAE\_MINIMUM\_TIME & \cellcolor{tabbg2}Specify the minimum amount of instrumentation time.\\
  \hline
\end{tabular}
}
\caption{Set of environment variables available to configure {\sf {E}xtrae}\ }
\end{table}
\par
\end{landscape}%
\lthtmlfigureZ
\lthtmlcheckvsize\clearpage}

{\newpage\clearpage
\lthtmlfigureA{landscape4570}%
\begin{landscape}
% latex2html id marker 4570

\par
\begin{table}
\centerline{
\small
\begin{tabular}{p{7.5cm} p{14cm}} 
  \hline
  {\bf Environment variable} & {\bf Description}\\
	\hline
  \cellcolor{tabbg2} EXTRAE\_MPI\_CALLER  & \cellcolor{tabbg2} Choose which MPI calling routines should be dumped into the tracefile.\\
  \cellcolor{tabbg1} EXTRAE\_MPI\_COUNTERS\_ON & \cellcolor{tabbg1} Set to 1 if MPI must report performace counter values.\\
  \cellcolor{tabbg2} & \cellcolor{tabbg2} Set to 1 if basic MPI statistics must be collected in burst mode\\
  \cellcolor{tabbg2} \multirow{-2}{*}{EXTRAE\_MPI\_STATISTICS} & \cellcolor{tabbg2} {\em (Only available in systems with Myrinet GM/MX networks)}.\\
  \cellcolor{tabbg1} EXTRAE\_NETWORK\_COUNTERS & \cellcolor{tabbg1}  Set to 1 to dump network performance counters at flush points.\\
  \cellcolor{tabbg2} EXTRAE\_PTHREAD\_COUNTERS\_ON & \cellcolor{tabbg2} Set to 1 if pthread must report performance counters values. \\
  \cellcolor{tabbg1} EXTRAE\_OMP\_COUNTERS\_ON & \cellcolor{tabbg1} Set to 1 if OpenMP must report performance counters values. \\
  \cellcolor{tabbg2} EXTRAE\_PTHREAD\_LOCKS & \cellcolor{tabbg2} Set to 1 if pthread locks have to be instrumented.\\
  \cellcolor{tabbg1} EXTRAE\_OMP\_LOCKS & \cellcolor{tabbg1} Set to 1 if OpenMP locks have to be instrumented.\\
  \cellcolor{tabbg2} EXTRAE\_ON & \cellcolor{tabbg2} Enables instrumentation\\
  \cellcolor{tabbg1} EXTRAE\_PACX\_CALLER & \cellcolor{tabbg1} Choose which PACX calling routines should be dumped into the tracefile.\\
  \cellcolor{tabbg2} EXTRAE\_PACX\_COUNTERS\_ON & \cellcolor{tabbg2} Set to 1 if PACX must report performace counter values.\\
  \cellcolor{tabbg1} EXTRAE\_PACX\_STATISTICS & \cellcolor{tabbg1} Set to 1 if basic PACX statistics must be collected in burst mode.\\
  \cellcolor{tabbg2} EXTRAE\_PROGRAM\_NAME & \cellcolor{tabbg2} Specify the prefix of the resulting intermediate trace files.\\
  \cellcolor{tabbg1} EXTRAE\_SAMPLING\_CALLER & \cellcolor{tabbg1} Determines the callstack segment stored through time-sampling capabilities.\\
  \cellcolor{tabbg2} & \cellcolor{tabbg2} Determines domain for sampling clock.\\
  \cellcolor{tabbg2} \multirow{-2}{*}{EXTRAE\_SAMPLING\_CLOCKTYPE} & \cellcolor{tabbg2} Options are: DEFAULT, REAL, VIRTUAL and PROF.\\
  \cellcolor{tabbg1} EXTRAE\_SAMPLING\_PERIOD & \cellcolor{tabbg1} Enable time-sampling capabilities with the indicated period.\\
  \cellcolor{tabbg2} EXTRAE\_SAMPLING\_VARIABILITY & \cellcolor{tabbg2} Adds some variability to the sampling period.\\
  \cellcolor{tabbg1} EXTRAE\_RUSAGE & \cellcolor{tabbg1} Instrumentation emits resource usage at flush points if set to 1.\\
  \cellcolor{tabbg2} EXTRAE\_SKIP\_AUTO\_LIBRARY\_INITIALIZE & \cellcolor{tabbg2} Do not automatically init instrumentation in the main symbol.\\
  \cellcolor{tabbg1} EXTRAE\_SPU\_DMA\_CHANNEL & \cellcolor{tabbg1} Choose the SPU-PPU dma communication channel.\\
  \cellcolor{tabbg2} EXTRAE\_SPU\_BUFFER\_SIZE & \cellcolor{tabbg2} Set the buffer size of the SPU side.\\
  \cellcolor{tabbg1} EXTRAE\_SPU\_FILE\_SIZE & \cellcolor{tabbg1} Set the maximum size for the SPU side (default: 5Mbytes).\\
  \cellcolor{tabbg2} EXTRAE\_TRACE\_TYPE & \cellcolor{tabbg2} Choose whether the resulting tracefiles are intended for Paraver or Dimemas.\\
  \hline
\end{tabular}
}
\caption{Set of environment variables available to configure {\sf {E}xtrae}\ ({\em continued})}
\end{table}
\par
\end{landscape}%
\lthtmlfigureZ
\lthtmlcheckvsize\clearpage}

\stepcounter{chapter}
\stepcounter{section}
\stepcounter{section}
\stepcounter{section}
\stepcounter{section}
\stepcounter{chapter}
\stepcounter{section}
\stepcounter{section}
\stepcounter{subsection}
\stepcounter{subsection}
\stepcounter{subsection}
\stepcounter{section}
\stepcounter{section}
\stepcounter{section}

\end{document}
